\documentclass[twoside,12pt]{wipb}

\usepackage{blindtext}
\usepackage{enumitem}
\usepackage{xcolor}

\katedra{Nazwa katedry}
\typpracy{in�ynierska}
%typpracy{magisterska}
\temat{Temat pracy in�ynierskiej / magisterskiej.}
\autor{Gal Anonim}
\promotor{dr in�. Doktor In�ynier}
\indeks{12345}
\studia{stacjonarne}
\rokakademicki{2013/2014}
\profil{studia II stopnia}
\kierunekstudiow{informatyka}
\specjalnosc{In�ynieria Oprogramowania}
\zakres{1. Zakres 1\newline 2. Zakres 2\newline 3. Zakres 3}

\hypersetup{
pdfauthor={Gal Anonim},
pdftitle={Praca in�ynierska},
pdfsubject={Temat pracy},
pdfkeywords={praca magisterska jakie� inne s�owa kluczowe},
pdfpagemode=UseNone,
linkcolor=black,
citecolor=black,
urlcolor=black
} 

\setlength{\epigraphwidth}{1\textwidth}

\begin{document}

\maketitle

\chapter*{\centering{\vspace{1in}Summary}}
\addcontentsline{toc}{chapter}{Streszczenie}
 
\epigraphhead[40]{
Subject of diploma thesis

Application of selected algorithms to optimize routes of municipal economy vehicles.}

Nowadays waste generated by residents is one of the problems that government have to deal with in Poland. Segregation and storage is not the only problem that can be encountered. Another difficulty that can be distinguished is their transport. The authors of the thesis proposed to conduct research to optimize vehicles routes acquired from the city of Bia{\l}ystok. Pawe{\l} Stypu{\l}kowski presented proposals for solving the problem using a genetic algorithm. On the other hand, Kamil {\L}{"e}towski explored the topic of route optimization using the ant algorithm in more detail. Where the final optimization solutions were presented by Przemys{\l}aw Noskowicz with greedy algorithms. After the theoretical introduction, the data sets on which the algorithms worked were presented. Next, the results that have been achieved with a particular solution are shown. There was also a list and comparison of the best achievements.


\cleardoublepage
{\setstretch{1.0}

{\hfill Za��cznik nr 4 do �Zasad post�powania przy przygotowaniu i obronie\par }
{\hfill pracy dyplomowej na PB�\par }

\hfill Bia�ystok, dnia 05.01.2020 r.
\\*

\noindent \textbf{Gal Anonim}

\noindent Imiona i nazwisko studenta

\noindent \textbf{12345}

\noindent Nr albumu

\noindent \textbf{informatyka, stacjonarne}

\noindent Kierunek i forma studi�w

\noindent \textbf{dr in�. Doktor In�ynier}

\noindent Promotor pracy dyplomowej
\\*
\\*

{\centering \textbf{O�WIADCZENIE}\par }


Przedk�adaj�c w roku akademickim 2019/2020 Promotorowi \textbf{dr in�. Doktor In�ynier} prac� dyplomow� pt.: \textbf{Temat pracy}, dalej zwan� prac� dyplomow�, \textbf{o�wiadczam, �e}: 

\begin{enumerate}
\setlength\itemsep{0em}
\item[1)] praca dyplomowa stanowi wynik samodzielnej pracy tw�rczej; 
\item[2)] wykorzystuj�c w pracy dyplomowej materia�y �r�d�owe, w tym w szczeg�lno�ci: monografie, artyku�y naukowe, zestawienia zawieraj�ce wyniki bada� (opublikowane, jak i nieopublikowane), materia�y ze stron internetowych, w przypisach wskazywa�em/am ich autora, tytu�, miejsce i rok publikacji oraz stron�, z kt�rej pochodz� powo�ywane fragmenty, ponadto w pracy dyplomowej zamie�ci�em/am bibliografi�;
\item[3)] praca dyplomowa nie zawiera �adnych danych, informacji i materia��w, kt�rych publikacja nie jest prawnie dozwolona;
\item[4)] praca dyplomowa dotychczas nie stanowi�a podstawy nadania tytu�u zawodowego, stopnia naukowego, tytu�u naukowego oraz uzyskania innych kwalifikacji;
\item[5)] tre�� pracy dyplomowej przekazanej do dziekanatu Wydzia�u Informatyki jest jednakowa w wersji drukowanej oraz w formie elektronicznej;
\item[6)] jestem �wiadomy/a, �e naruszenie praw autorskich podlega odpowiedzialno�ci na podstawie przepis�w ustawy z dnia 4 lutego 1994 r. o prawie autorskim i prawach pokrewnych (Dz. U. z 2019 r. poz. 1231, p�n. zm.), jednocze�nie na podstawie przepis�w ustawy z dnia 20 lipca 2018 roku Prawo o szkolnictwie wy�szym i nauce (Dz. U. poz. 1668, z p�n. zm.) stanowi przes�ank� wszcz�cia post�powania dyscyplinarnego oraz stwierdzenia niewa�no�ci post�powania w sprawie nadania tytu�u zawodowego;
\item[7)] udzielam Politechnice Bia�ostockiej nieodp�atnej, nieograniczonej terytorialnie i czasowo licencji wy��cznej na umieszczenie i przechowywanie elektronicznej wersji pracy dyplomowej w zbiorach systemu Archiwum Prac Dyplomowych Politechniki Bia�ostockiej oraz jej zwielokrotniania i udost�pniania w formie elektronicznej w zakresie koniecznym do weryfikacji autorstwa tej pracy i ochrony przed przyw�aszczeniem jej autorstwa.  
\end{enumerate}
 
\hfill .���������������.
\begin{flushright}
czytelny podpis studenta	
\end{flushright}
}          


%\biblioteka{}

\pagestyle{plain}

\setcounter{tocdepth}{2}
\tableofcontents

\chapter*{Wst�p -- KL, PN, PS}
\addcontentsline{toc}{chapter}{Wst�p}
  
	Odpady generowane przez mieszka�c�w s� jednym z problem�w z jakimi w�adze musz� sobie radzi� w Polsce. Nie jest to problem jedynie wielkich metropolii, ale r�wnie� mniejszych miasteczek. Ilo�� produkowanych �mieci jest olbrzymia, a to r�wnie� powoduje kolejne problemy. Segregacja oraz sk�adowanie to nie jest jedyny problem z jakim mo�na si� spotka�. Kolejnym utrudnieniem jaki mo�na wyr�ni� jest ich transport.

	W 2018 roku w Bia�ymstoku zosta� zorganizowany 24-godzinny Hackaton Miejski w Bia�ostockim Parku Naukowo-Technologicznym. Organizatorzy tego przedsi�wzi�cia przygotowali du�� ilo�� merytorycznych materia��w na temat zbi�rki odpad�w komunalnych w mie�cie Bia�ystok. Uczestnicy hackatonu zauwa�yli wiele problem�w z jakimi boryka si� miasto. Jednak najwi�ksz� uwag� organizator�w przyku�y r�ne anomalie.
	
	Jedna z grup zauwa�y�a, �e trasy jakie pokonuj� ci�ar�wki s� nieoptymalne. Jako przyk�ad zosta�a przedstawiona droga losowej ci�ar�wki. Mo�na by�o zauwa�y�, �e przebyta �cie�ka przecina si� w wielu miejscach. Na podstawie tylko jednego przyk�adu mo�na doj�� do wniosku, �e problem jest globalny i mo�e dotyczy� wszystkich tras. Organizatorzy przedsi�wzi�cia przyznali, �e nie przyk�adali do tego aspektu uwagi, ale mo�e to by� kluczowe w oszcz�dno�ci czasu oraz pieni�dzy.
	
	Przy pomocy komputer�w oraz odpowiednio napisanego programu mo�na wyznaczy� �cie�ki, kt�re b�d� spe�nia� za�o�enia biznesowe oraz b�d� odpowiednio optymalne. Wyzwaniem jest jedynie znalezienie odpowiedniego rozwi�zania kt�re nada prawid�owy kierunek przeprowadzanym optymalizacjom. Na przestrzeni lat zosta�o opracowanych wiele rozwi�za�. Niestety, nie ka�de rozwi�zanie jest w stanie poradzi� sobie z konkretnymi danymi w ten sam spos�b.
	
	Celem pracy jest zastosowanie i por�wnanie algorytm�w optymalizacji w celu znalezienia najkr�tszych tras przejazdu ci�ar�wek z odpadami komunalnymi. Dla algorytm�w zostan� znalezione takie parametry wej�ciowe, aby uzyska� jak najkr�tsz� tras�. Po przeprowadzeniu bada�, rozwi�zania zostan� por�wnane pod k�tem optymalizacji oraz szybko�ci dzia�ania.
	
	W rozdziale pierwszym zostanie przedstawiony og�lny zarys problemu. Po wprowadzeniu do problemu zostan� om�wione metody optymalizacji tras przejazd�w. 
	
	W rozdziale drugim Pawe� Stypu�kowski (PS) przedstawi algorytm genetyczny. W tym rozdziale mo�na znale�� informacje, w jaki spos�b dzia�a to podej�cie. Dog��bnie zostan� opisane sk�adowe tego algorytmu. W rozdziale zostan� opisane parametry wej�ciowe, metody krzy�owania i metody mutacji.
	
	Kamil ��towski (KL) w rozdziale trzecim zaproponuje optymalizacj� tras przy pomocy algorytmu mr�wkowego. Opr�cz wprowadzenia do algorytmu zosta� zawarty opis dzia�ania. Autor swoj� uwag� skupi� r�wnie� na bardzo wa�nej sk�adowej tego rozwi�zania, jakim s� feromony wykorzystywane przez mr�wki.
	
	Algorytmy zach�anne to rozwi�zanie, jakie zaproponuje Przemys�aw Noskowicz (PN) w rozdziale czwartym. Autor tego rozdzia�u opisze kilka wybranych metod optymalizacji. W�r�d nich mo�na wymieni� algorytm najbli�szego s�siada, algorytm najmniejszej kraw�dzi, algorytm A*. Na koniec za pomoc� metody 2-opt oraz zaaplikowaniu wsp�czynnika b��du utworzone trasy zosta�y poprawione.
	
	Po teoretycznym wst�pie w rozdziale pi�tym zosta�y przedstawione zbiory danych na kt�rych zostan� przeprowadzone badania. Opisane b�d� r�wnie� problemy z przetworzeniem zbior�w danych.
	
	W sz�stym rozdziale zostan� przedstawione wyniki tych bada� jakie uda�o si� osi�gn�� dla poszczeg�lnych zbior�w danych. W tym rozdziale zostan� por�wnane najlepsze osi�gni�cia dla poszczeg�lnych algorytm�w.
	
	Ostatni rozdzia� b�dzie podsumowaniem pracy. Zostan� w nim r�wnie� przedstawione najlepsze wyniki oraz potencjalne mo�liwe kierunki rozwoju bada�.
	
	
\chapter{Og�lny problem}

Badany przez nas problem jest w informatyce nazywany problemem komiwoja�era. W tym rozdziale zostanie on om�wiony oraz metody optymalizacji.

\section{Problem komiwoja�era} 
Problem komiwoja�era (ang. travelling salesman problem - TSP) nale�y do rodziny problem�w NP-trudnych. Znalezienie najlepszego rozwi�zania jest trudne i fascynuje naukowc�w od wielu lat. Niekt�rzy poddaj� pod w�tpliwo�� znalezienie efektywnego rozwi�zania, czyli takiego kt�rego czas dzia�ania jest maksymalnie wielomianowy. Aktualnie istnieje wiele rozwi�za� tego problemu, a proponowane podej�cia s� bardzo interesuj�ce. Niekt�re z nich bazuj� na lokalnych przeszukiwaniach grafu, a inne opieraj� si� na przyk�adach kt�re wyst�puj� w przyrodzie.

Podobnym problemem do TSP jest problem konika szachowego. Problem ten r�wnie� nale�y do rodziny problem�w NP-zupe�nym. Ju� w XVIII wieku badania nad tym problemem rozpocz�� Euler. Rozwi�zanie tego problemu polega na znalezieniu �cie�ki jak� ma przeby� konik szachowy, tak aby odwiedzi� ka�de pole na szachownicy tylko i wy��cznie raz. Skoczek porusza si� po planszy zgodnie z okre�lonym ruchem, a plansza szachowa mo�e mie� r�ny rozmiar. Konik porusza si� a� do momentu odwiedzenia wszystkich p�l lub do momentu w kt�rym nie ma mo�liwo�ci odwiedzenia kolejnego pola.

Optymalizacja tras od zawsze jest obecna w historii ludzko�ci. Nawet takie trywialne problemy jak podr� pomi�dzy 3 miejscowo�ciami mo�e zosta� sklasyfikowany jako problem komiwoja�era. Chocia� dok�adne wskazanie na �r�d�o problemu TSP nie jest znane, to ju� w 1832 roku w przewodniku dla podr�uj�cych po Niemczech i Szwajcarii zosta�a zawarta informacja o optymalizacji trasy przejazdu. Nie ma tam zawartych �adnych teorii matematycznych w zwi�zku z czym nie mo�na uzna� tego dzie�a za pocz�tek rozwa�a� nad problemem komiwoja�era.\cite{kl_poczatek_rozwazan}

W XIX wieku William Hamilton stworzy� fundamenty pod definicj� TSP. W rozwi�zaniu problemu komiwoja�era nale�y znale�� cykl w grafie. W sk�ad takiego cyklu musi zosta� zawarty ka�dy z wierzcho�k�w. Ka�dy z wierzcho�k�w mo�e znajdowa� si� w rozwi�zaniu dok�adnie tylko raz. Cykl kt�ry spe�nia wymieniony warunek jest cyklem Hamiltona. Je�li w grafie mo�na wyr�ni� cykl z opisanymi powy�ej warunkami, to graf jest grafem Hamiltonowskim. 

Na \ref{kl_graf_nie_hamiltona} zosta� przedstawiony graf bez cyklu Hamiltona. W grafie tym nie mo�na znale�� takiego po��czenia kt�re zawiera wszystkie wierzcho�ki przechodz�c przez ka�dy z nich dok�adnie raz. Istnieje mo�liwo�� przej�cia przez wszystkie wierzcho�ki jedynie po powt�rnym odwiedzeniu przynajmniej jednego wierzcho�ka.

\begin{figure}[h]
	\centering
	\includegraphics[scale=0.75]{grafika/kl_graf_nie_hamiltona.png}
	\caption{Graf bez cyklu Hamiltona.}
	\label{kl_graf_nie_hamiltona}
\end{figure}

Graf z \ref{kl_graf_hamiltona} posiada po��czenie kraw�dzi dzi�ki kt�remu mo�na przej�� po wszystkich wierzcho�kach dok�adnie raz. Takie przej�cie jest w�a�nie cyklem Hamiltona w zwi�zku z czym graf jest Hamiltonowski. Wyruszaj�c przyk�adowo z punktu F mo�emy przej�� kolejno do E - G - D - B - A - C. W ten spos�b odwiedzimy wszystkie wierzcho�ki tylko raz. 

\begin{figure}[h]
	\centering
	\includegraphics[scale=0.75]{grafika/kl_graf_hamiltona.png}
	\caption{Graf z cyklem Hamiltona.}
	\label{kl_graf_hamiltona}
\end{figure}

W latach 30 XX wieku Merrill Meeks Flood rozpocz�� rozwa�ania nad optymalizacj� przejazdu autobus�w szkolnych. Dzia�alno�� t� mo�emy uzna� za pocz�tek pracy nad problemem TSP. Wraz z up�ywem czasu zainteresowanie problemem optymalizacyjnym narasta�o, a co za tym idzie powstawa�y nowe pomys�y na algorytmy. Jednak �aden z pomys��w nie jest w stanie zaproponowa� dok�adnego rozwi�zania kt�re jest w stanie przedstawi� rezultat w czasie wyra�onym za pomoc� wielomianu.

\section{Metody optymalizacji} 
	
Jednym z proponowanych rozwi�za� jest algorytm Helda Karpa kt�ry jest oparty na programowaniu dynamicznym. Z�o�ono�� pami�ciowa tego algorytmy wynosi O(n razy 2 do n), a czasowa O(n do 2 razy 2 do n). W algorytmie tym na ka�dym kroku wyznaczamy punkt kt�ry powinien by� przedostatni na trasie. Aby wyznaczy� poprzednika nale�y skorzysta� ze wzoru w kt�rym poszukiwana jest najmniejsza warto�� pomi�dzy punktami.

Innym przyk�adem , kt�ry mo�na wykorzysta� do rozwi�zania problemu komiwoja�era jest algorytm najbli�szego s�siada. Rozwi�zanie to wykorzystuje strategi� zach�ann�. W algorytmie szukamy aktualnie najlepszego ruchu. W tym celu przeszukiwani s� jedynie s�siedzi kt�rzy s� najbli�ej aktualnego punktu. Z�o�ono�� takiego algorytmu jest szacowana na O(n do 2).

Opr�cz standardowych przeszukiwa� zbior�w na przestrzeni lat pojawi�y si� propozycje kt�re wprowadzaj� elementy losowo�ci. Przyk�adem takiego rozwi�zania mog� by� algorytm genetyczny oraz algorytm mr�wkowy. Nale�� one do grup algorytm�w heurystycznych, czyli do takich, kt�re nie daj� gwarancji znalezienia dok�adnego rozwi�zania.

W pracy zostan� zbadane rozwi�zania problemu za pomoc� algorytmu genetycznego, mr�wkowego oraz zach�annego. W pozosta�ych podrozdzia�ach zostan� opisane ich klasyczne wersje.

\subsection{Algorytm genetyczny - Pawe�}
Algorytm genetyczny (ang.  \textit{GA - Genetic Algorithm}) polega na symulacji proces�w genetycznych zachodz�cych w populacjach osobnik�w, stosuje si� je g��wnie przy zadaniach optymalizacyjnych. W przyrodzie wi�kszo�� gatunk�w od wiek�w, w tym przede wszystkim i cz�owiek, w kolejnych swoich pokoleniach si� rozwin�o i dostosowa�o do otaczaj�cych warunk�w na �wiecie. Gdy jaki� osobnik urodzi si� z cech�, kt�ra jest przydatna w przetrwaniu, przeka�� t� cech� kolejnym pokoleniom. Najs�absze osobniki w populacji maj� zar�wno mniejsze szanse na przetrwanie oraz rozmno�enie, czyli przekazanie swoich cech potomkom. W przyrodzie zazwyczaj silniejsi wygrywaj�. W latach 50 XX wieku zacz�to symulowa� te procesy w informatyce. W latach 60 John Henry Holland zastosowa� algorytm genetyczny przy pracach nad systemami adaptacyjnymi, a 1975  wyda� ksi��k� \textit{Adaptation in Natural and Artificial Systems}, w kt�rej to opisa�.\cite{genetic_1}. Algorytm genetyczny poszukuje najlepsze rozwi�zania, w�r�d populacji potencjalnych rozwi�za�. Jest to g��wna cecha, kt�ra odr�nia go od tradycyjnych metod optymalizacji. Ka�de rozwi�zanie ulega ocenie na podstawie jego dopasowania do problemu - funkcja przystosowania. Algorytm na populacji symuluje zjawiska ewolucyjne, krzy�uj�c i mutuj�c rozwi�zania, stosuj�c probabilistyczne regu�y wyboru. W ka�dym takim nowym pokoleniu najs�absi s� usuwani, wi�c w kolejnych etapach populacja sk�ada si� z coraz to lepszych rozwi�za� \cite{genetic_9}. Kolejne pokolenia s� generowane, a� zostanie spe�niony warunek zako�czenia. Mo�e to by� z g�ry ustalony czas trwania, ilo�� kolejnych pokole� lub brak poprawy w�r�d nowych rozwi�za�.

Algorytmy genetyczne i jego odmiany zrewolucjonizowa�y systemy informatyczne. Nie s� one algorytmami, kt�re wyliczaj� dok�adne wyniki, ale przy odpowiedniej implementacji i ustaleniu parametr�w wej�ciowych, pozwalaj� osi�gn�� dobre rezultaty. Bardzo wa�ny jest czas znalezienia takiego rozwi�zania. Je�li rozwi�zanie idealne obliczane jest przez algorytm tradycyjny w ci�gu 24 godziny, a algorytm genetyczny w trakcie kilku minut znajdzie rozwi�zanie b�d�ce w s�siedztwie, swoj� efektywno�ci� wygra ten drugi.  U�ytkownik nie b�dzie chcia� czeka� ca�ej doby na wynik swojego zapytania. Oczywi�cie, algorytmy genetyczne te� mog� znale�� najlepsze rozwi�zanie. Kwesti� ograniczenia jest zawsze czas.

Medycyna jest jedn� z wa�niejszych dziedzin, gdzie wykorzystuje si� algorytmy genetyczne. Zbiory danych i przestrze� przeszukiwa� jest ogromna i z�o�ona. Zazwyczaj w oparciu o te informacj�, lekarz musi podj�� decyzj�, czy np. nowotw�r jest z�o�liwy, czy ma �agodny przebieg. Algorytm genetyczny pozwala wspom�c lekarza przy podejmowania takich decyzji, przetwarzaj�c i analizuj�c te ogromne zbiory. \cite{genetic_2}. Kolejn� ga��zi� gospodarki, w kt�rych zastosowanie znajduje algorytm genetyczny, jest przemys� spo�ywczy, a konkretnie optymalizacja linii produkcyjnych. Za ich pomoc� algorytmu genetycznego wyznacza si� parametry takie jak temperatura, ci�nieniu lub zapotrzebowanie mocy. Dzi�ki temu wszystkie procesy technologiczne zachodz�ce w maszynach i urz�dzeniach s� wydajne i zoptymalizowane\cite{genetic_4}. Algorytmy genetyczne cz�sto stosuje si� jako  wskazanie punkt�w pocz�tkowych w innych metodach optymalizacyjnych. Poza podanymi przyk�adami, znajduj� one zastosowanie praktycznie wsz�dzie: ekonomia, gie�da, przemys� lotniczy, kombinatoryka, sieci komputerowe, zarz�dzanie �a�cuchem dostaw, a tak�e ustalanie czasu reklamy w telewizji \cite{genetic_5}.



\subsection{Algorytm mr�wkowy}

Obserwacje nad zachowaniami w przyrodzie wielokrotnie mia�y wp�yw na rozw�j nowych rozwi�za�. Tak jak w przypadku algorytmu genetycznego, tak i w przypadku algorytmu mr�wkowego pomys� zosta� zaczerpni�ty z przyrody. Dok�adne dzia�anie algorytmu mr�wkowego wzoruje si� na zachowaniu kolonii mr�wek. Dzi�ki pracy zespo�owej, owady te s� w stanie wypracowa� optymaln� �cie�k� mi�dzy siedliskiem a znalezionym pokarmem.

Dla niejednego gatunku problematyczne mog�oby by� odtworzenie przebytej �cie�ki. Na pocz�tku nale�a�oby zada� sobie pytanie w jaki spos�b te niewielkich rozmiar�w owady s� w stanie znacz�co u�atwi� sobie przetrwanie? Istotn� rol� odgrywa tutaj wspomniana ju� praca zespo�owa. To dzi�ki niej mr�wki s� w stanie optymalizowa� tras�. Innym wa�nym czynnikiem determinuj�cym popraw� �cie�ki jest zapach jaki zostawiaj� mr�wki. 

Zapach nie jest niczym innym jak feromonem wytwarzanym przez mr�wki. Dzi�ki pozostawionemu zapachowi mr�wki wiedzia�y w jaki spos�b poruszali si� ich poprzednicy w zwi�zku z czym odtworzenie trasy nie stanowi�o ju� powa�nego problemu. Przy kolejnych iteracjach kolonia pr�buje optymalizowa� aktualn� tras�. W tym celu r�wnie� wykorzystuje zapach pozostawiony w poprzednich przej�ciach. �cie�ka jest losowo zmieniana w celu optymalizacji. Je�li modyfikacja przynios�a oczekiwany efekt, to trasa zostaje zmieniona. 

Feromony s� istotnym czynnikiem w ca�ym procesie. To dzi�ki nim trasa jest ulepszana. Zapach posiada jedn� z charakterystyk kt�ra na pocz�tku mo�e wydawa� si� problematyczna. Wraz z up�ywem czasu si�a zapachu s�abnie a� do ca�kowitego zanikni�cia. W�a�ciwo�� ta jest zalet�, a nie wad�. To dzi�ki pracy zespo�owej zapach na najlepszej trasie jest podtrzymywany, a na s�abszych zanika. Dzi�ki tej selekcji d�u�sze trasy nie s� brane pod uwag� w wyniku czego zostaje trasa najkorzystniejsza.

Do wyznaczenia optymalnej trasy potrzebne s� d�ugo�ci jakie nale�y przeby� do przemieszczania si� mi�dzy punktami. W \ref{tabela_kosztow_ant} przedstawione s� przyk�adowe odleg�o�ci.

\begin{table}[htb]
\centering
\caption[Kr�tki podpis tabeli 1 -- do spisu tre�ci]{Warto�ci koszt�w}
\begin{tabular}{|c|c|c|c|c|}
\hline
  & A & B & C & D \\\hline
A & 0 & 3 & 8 & 2 \\\hline
B & 3 & 0 & 2 & 4 \\\hline
C & 8 & 2 & 0 & 1 \\\hline
D & 2 & 4 & 1 & 0 \\\hline
\end{tabular}
\label{tabela_kosztow_ant}
\end{table}

\begin{figure}[h]
	\centering
	\includegraphics[scale=0.75]{grafika/kl_init_trasa_alg.png}
	\caption{Graf z wagami}
	\label{fig:kl_init_trasa_alg}
\end{figure}

R�wnie wa�ne jest wyznaczenie pocz�tkowych wsp�czynnik�w feromon�w. Na pocz�tku nadajmy wszystkim kraw�dziom w grafie warto�ci r�wne 1, a wsp�czynnik parowania niech wynosi 0.5.

\begin{table}[htb]
\centering
\caption[Kr�tki podpis tabeli 1 -- do spisu tre�ci]{Warto�ci feromon�w}
\begin{tabular}{|c|c|c|c|c|}
\hline
  & A & B & C & D \\\hline
A & 0 & 1 & 1 & 1 \\\hline
B & 1 & 0 & 1 & 1 \\\hline
C & 1 & 1 & 0 & 1 \\\hline
D & 1 & 1 & 1 & 0 \\\hline
\end{tabular}
\label{tabela_feromonow_ant}
\end{table}

\begin{figure}[h]
	\centering
	\includegraphics[scale=0.75]{grafika/kl_init_p_alg.png}
	\caption{Graf z pocz�tkowymi warto�ciami feromon�w}
	\label{fig:kl_init_trasa_alg}
\end{figure}

Posiadaj�c pocz�tkowe dane wyznaczmy w spos�b losowy trasy dla dw�ch agent�w: L1 i L2. Agent L1 porusza� si� tras� w kt�rej odwiedzi� wierzcho�ki w nast�puj�cej kolejno�ci: A, B, C, D, A. 

\begin{figure}[h]
	\centering
	\includegraphics[scale=0.75]{grafika/kl_l1_trasa_alg.png}
	\caption{Trasa przebyta przez agenta L1}
	\label{fig:kl_init_trasa_alg}
\end{figure}

Agent L2 wyznaczy� nast�puj�c� tras�: A, C, B, D, A. 

\begin{figure}[h]
	\centering
	\includegraphics[scale=0.75]{grafika/kl_l2_trasa_alg.png}
	\caption{Trasa przebyta przez agenta L2}
	\label{fig:kl_init_trasa_alg}
\end{figure}

Mr�wki odpowiednio pokona�y dystans 8 i 16 punkt�w. Dzi�ki tej informacji mo�na zaktualizowa� warto�ci feromon�w na poszczeg�lnych kraw�dziach. Do oblicze� wykorzystany jest wz�r: TUTAJ WZ�R DODA�. Kraw�dzie A-B i C-D zosta�y odwiedzona jedynie przez agenta L1 w wyniku czego warto�� feromonu zostaje zmieniona na 10/16. Nast�pnie kraw�dzie A-C i B-D zostaj� zaktualizowane na 9/16. Kraw�dzie A-D i B-C s� odwiedzone dwukrotnie a warto�� feromon�w wynosi 11/16.

\begin{figure}[h]
	\centering
	\includegraphics[scale=0.75]{grafika/kl_p_alg.png}
	\caption{Graf z warto�ciami feromon�w po modyfikacji}
	\label{fig:kl_init_trasa_alg}
\end{figure}

Ostatni� faz� algorytmu jest wyznaczenie prawdopodobie�stwa z jakim kolejni agenci b�d� wybiera� kolejny wierzcho�ek. W kolejnej iteracji agent L3 znajduje si� w wierzcho�ku B. Do wyboru ma kraw�dzie A, C i D. Wed�ug wzoru TUTAJ WZ�R DODA� wyliczane jest prawdopodobie�stwo dla wszystkich mo�liwo�ci. Przej�cie z kraw�dzi B do kraw�dzi A wynosi oko�o 20 PROCENT, do kraw�dzi D 30 procent, a do kraw�dzi C oko�o 50 PROCENT. 

Optymalna trasa zostanie wykszta�cona po wykonaniu wielu iteracji. Ilo�� iteracji nie jest zdefiniowana i dla ka�dego przypadku mo�e by� r�na. Sytuacja wygl�da identycznie w przypadku wyboru warto�ci p. Wa�ne jest natomiast to, aby w trakcie dzia�ania algorytmu nie modyfikowa� tej warto�ci. Powinno ona by� taka sama na ka�dym kroku algorytmu.

\subsection{Algorytmy zach�anne}
	Kolejnym spojrzeniem na poruszany w pracy problem komiwoja�era s� algorytmy zach�anne (ang. greedy algorithms). Nie znajdziemy dowodu na to czy dla podanego problemu algorytm zach�anny znalaz� poprawny wynik, jednak stosuj�c si� do pewnych zasad mo�emy okre�li�, �e dla naszego problemu istnieje rozwi�zanie zach�anne. G��wn� strategi� jak� si� kieruj� jak sama nazwa wskazuje jest dokonywanie wybor�w, kt�re w danej chwili wydaj� si� najlepsze. Oznacza to, �e dokonuje si� wybor�w optymalnych lokalnie w nadziei, �e te wybory doprowadz� do rozwi�zania globalnego w rozs�dnym czasie. W odr�nieniu do strategii zastosowanej w programowaniu dynamicznym wybory podejmowane przez algorytmy zach�anne nie s� uzale�nione od wybor�w przesz�ych. Kolejnym kryterium stosowanym do ocenienia poprawno�ci rozwi�zania zach�annego jest w�asno�� optymalnej pod struktury, m�wi�ca o tym, �e optymalne rozwi�zanie dla ca�ego problemu istnieje jedynie przy optymalnym rozwi�zaniu pod problem�w. Dane kryterium jest r�wnie� istotne w przypadku rozwi�zywania problem�w metod� programowania dynamicznego. Algorytmy zach�anne nie zawsze prowadz� jednak do optymalnych rozwi�za�, jednak�e dla w wi�kszo�ci problem�w daj� wystarczaj�ce rezultaty.
	Skorzystanie z algorytm�w zach�annych cz�sto okazuje si� niewystarczaj�ce. Aby uzyska� lepszy efekt i polepszy� zbudowane ju� trasy mo�emy wykorzysta� algorytmy lokalnej optymalizacji (ang. local search). U�ycie ich na zwr�conych przez algorytmy zach�anne trasach powinno zminimalizowa� odleg�o�ci mi�dzy wierzcho�kami w celu poprawienia otrzymanego rozwi�zania. Dok�adniejszy opis dzia�ania wybranych algorytm�w zach�annych i metod optymalizuj�cych zosta� przedstawiony w podrozdzia�ach.
	


\chapter{Algorytm genetyczny - Pawe�}
	
\section{Chromosom}
	Chromosom z definicji jest to ci�g gen�w reprezentuj�cy dane rozwi�zanie. Z kolei gen przenosi informacj� o cechach. Mo�liwo�� osi�gni�cia sukcesu w algorytmie genetycznym jest tylko wtedy, gdy odpowiednio zakoduje si� cechy i ustali funkcj� przystosowania. Do zakodowania badanego problemu zostanie u�yta metody permutacyjna, gdzie ka�dy punkt musi zosta� odwiedzony tylko raz. Ka�demu punktowi przed wylosowanie tras zostanie przypisany unikalny indeks, b�dzie on odpowiada� genowi. Nast�pnie dla ka�dego z $N$ osobnik�w zostanie zapisany chromosom w postaci ci�gu permutacyjnego. Na rysunku \ref{kodowanie} zosta�y zilustrowane przyk�ady kodowania permutacyjnego. Nale�y zwr�ci� uwag�, �e pocz�tek nie jest wa�ny, wiec chromosom 1-2-3-4-5-6-7 b�dzie taki sam jak 5-6-7-1-2-3-4. Ka�dy chromosom musi zawiera� wszystkie geny.
	
\begin{figure}[h]
	\centering
	\includegraphics[scale=0.75]{grafika/chromosom.png}
	\caption{Kodowanie chromosom�w}
	\label{kodowanie}
\end{figure}
	
\section{Funkcja przystosowania}
	Algorytm genetyczny z definicji szuka osobnika z najwi�ksz� warto�ci� funkcji przystosowania. W badanym problemie nale�y znale�� najkr�tsz� tras�. W momencie, gdy d�ugo�� takiej trasy zostanie odwr�cona, to oka�e si�, �e im wi�ksza warto�� odwrotno�ci tym kr�tsza trasa. Zatem wz�r funkcji przystosowania to \[ fp = \frac{1}{s+1}\]	gdzie s - d�ugo�� trasy, czyli suma odleg�o�ci pomi�dzy genami(wierzcho�kami) 1-2-3-4-5-6-7.

\section{Metody krzy�owania}
	W pracy zostan� zbadane trzy rodzaje metod krzy�owania. Ka�de z nich charakteryzuje si� czym� innym je�li chodzi o liczb� potomk�w oraz porz�dek gen�w wzgl�dem rodzic�w. Geny mog� by� na takich samych pozycjach jak u przodk�w lub w ca�kiem innym porz�dku.
\subsection{Krzy�owanie z cz�ciowym odwzorowaniem - PMX}
	PMX(ang. partially mapped crossover) jest odmian� krzy�owania dwupunktowego w kt�rym powstaje dw�jka potomstwa. Losowane s� dwa dowolne punkty w kt�rych rozcina si� rodzic�w. Po czym wyci�te segmenty zamienia si�  miejscami(rys. \ref{krzyzowanie1}). Powsta�e w ten spos�b chromosomy nie s� permutacj� poniewa� zawieraj� powt�rzenia. Nale�y je zamieni� na te geny, kt�re zosta�y wyci�te. W tym celu nast�puje okre�lenie relacji pomi�dzy genami w wyci�tych segmentach, kt�re nast�pnie si� zamienia ze sob� w chromosomach(rys. \ref{krzyzowanie2}). Tak powsta�e dzieci posiadaj� ju� struktur� permutacyjn�.
	
\begin{figure}[h]
	\centering
	\includegraphics[scale=0.75]{grafika/krzyzowanie1.png}
	\caption{Krzy�owanie PMX cz�� 1}
	\label{krzyzowanie1}
\end{figure}

\begin{figure}[h]
	\centering
	\includegraphics[scale=0.75]{grafika/krzyzowanie2.png}
	\caption{Krzy�owanie PMX cz�� 2}
	\label{krzyzowanie2}
\end{figure}

\subsection{Krzy�owanie z zachowaniem porz�dku - OX}
	OX(ang. order crossover) jest r�wnie� odmian� krzy�owania dwupunktowego. W przeciwie�stwie do PMX wynikiem b�dzie tylko jedno dziecko. Pierwszym krokiem jest wylosowanie dw�ch dowolnych punkt�w w kt�rych zostanie rozci�ty pierwszy rodzic. Nast�pnie w drugim rodzicu nale�y usun�� punkty, kt�re znajduj� si� w wyci�tym segmencie pierwszego rodzica. Ten wyci�ty segment nale�y wklei� w to samo miejsce w rodzicu drugim (rys. \ref{krzyzowanie2}). Tak powsta�y chromosom jest potomkiem. Na pierwszy rzut oka mo�na zauwa�y�, �e w dziecku geny, kt�re zosta�y wyci�te z pierwszego rodzica znajduj� si� na tych samych miejscach, pozosta�e natomiast si� przemie�ci�y.
	
\begin{figure}[h]
	\centering
	\includegraphics[scale=0.75]{grafika/krzyzowanie3.png}
	\caption{Krzy�owanie OX}
	\label{krzyzowanie2}
\end{figure}

\subsection{Krzy�owanie cykliczne - CX}
	CX(ang. cycle crossover) w przeciwie�stwie do PX i OX nie polega na krzy�owaniu w dw�ch okre�lonych punktach. Aby wyznaczy� potomka, nale�y w dowolnym rodzicu znale�� cykl permutacji, zaczynaj�c od dowolnego miejsca. Polega to na kopiowaniu gen�w z jednego rodzica wed�ug pozycji okre�lonych przez rodzica drugiego. Po wyznaczeniu cyklu, pozosta�e geny s� kopiowane z drugiego chromosomu. W tak powsta�ym dziecku wszystkie geny zajmuj� tak� sam� pozycj� jak w kt�rym� z rodzic�w, inaczej ni� to by�o przy OX.

\begin{figure}[h]
	\centering
	\includegraphics[scale=0.75]{grafika/krzyzowanie4.png}
	\caption{Krzy�owanie CX}
	\label{krzyzowanie3}
\end{figure}

\section{Mutacje}
	W pracy zostan� zbadane trzy r�ne rodzaje mutacji. S� to specjalne mutacje wykorzystywane przy strukturach permutacyjnych. Ka�da z nich zostanie r�wnie� zbadana z r�n� warto�ci� pm. Z regu�y nie mo�e by� one du�e, aby nie niszczy� rozwi�za�. Powinno delikatnie wprowadza� r�norodno��, aby by�a mo�liwo�� odkry� w nowych obszarach. Bardzo wa�ne jest, aby zmiany nie zaburza�y struktury permutacyjnej chromosomu.
	
\subsection{Mutacja odwracaj�ca}
	W mutacji odwracaj�cej wybierany jest dowolny podci�g gen�w z chromosomu, a nast�pnie ich kolejno�� jest odwracana(rys. \ref{mutacja1}).
\begin{figure}[h]
	\centering
	\includegraphics[scale=0.75]{grafika/mutacja1.png}
	\caption{Mutacja odwracaj�ca}
	\label{mutacja1}
\end{figure}
\subsection{Mutacja wstawiaj�ca}
	W tej mutacji dowolny gen jest przestawia si� w losowe miejsce(rys. \ref{mutacja2}).
\begin{figure}[h]
	\centering
	\includegraphics[scale=0.75]{grafika/mutacja2.png}
	\caption{Mutacja wstawiaj�ca}
	\label{mutacja2}
\end{figure}
\subsection{Mutacja zamieniaj�ca}
 	W mutacji zamieniaj�cej zamienia si� dwa dowolne geny miejscami(rys. \ref{mutacja3}).
 \begin{figure}[h]
	\centering
	\includegraphics[scale=0.75]{grafika/mutacja3.png}
	\caption{Mutacja wstawiaj�ca}
	\label{mutacja3}
\end{figure}

	
\chapter{Algorytm mr�wkowy}

S�owo rozw�j mo�e by� zestawiane z wieloma rzeczownikami. Wiele dziedzin ci�gle si� rozwija, powstaj� nowe udogodnienia kt�re wp�ywaj� na wiele dziedzin �ycia. Dzi�ki rozwojowi techniki ludzie s� w stanie osi�ga� cele kt�re jaki� czas temu mog�y by� tylko marzeniami. Rozw�j techniki r�wnie� potrzebuje inspiracji do tworzenia nowych, lepszych rozwi�za�

Szybko�� oraz dok�adno�� rozwoju zale�y od wielu czynnik�w. Dzi�ki pracy zespo�owej pewne problemy mog� by� rozwi�zywane szybciej i dok�adniej. Wysi�ek w�o�ony przez grup� procentuje szybko, a same efekty mog� by� r�wnie� satysfakcjonuj�ce. Istotnym czynnikiem jest wysi�ek wk�adany przez ka�dego cz�onka grupy.

Obserwuj�c przyrod� mo�emy zauwa�y� w jaki spos�b zwierz�ta radz� sobie z r�nymi problemami. Z�o�one grupy mog� by� spokojniejsze o zdobycie po�ywienia czy te� o przetrwanie w ci�kich warunkach. Praca zespo�owa jest jedn� z cech kt�rej osobniki w grupie ucz� si� nie b�d�c nawet tego do ko�ca �wiadomym.

\section{Wprowadzenie do algorytmu}

Na przestrzeni czasu wiele gatunk�w zwierz�t �yj�cych na ziemi przystosowa�o si� do panuj�cych tu warunk�w. Jedn� z takich gatunk�w s� mr�wki. Te niewielkich rozmiar�w owady posiadaj� zdolno�ci pomagaj�ce im przetrwa� w�r�d najci�szych warunk�w. Mr�wki �yj� w stadach w zwi�zku z czym wykorzystuj� prac� zespo�ow� do rozwi�zywania problem�w jakie codziennie napotykaj� na swojej drodze.

Aby zapewni� przetrwanie stadu mr�wki potrzebuj� zapewni� sobie dost�p do pokarmu. Dziesi�tki tysi�cy mr�wek maj� swoje schronienie w mrowiskach. To tam trafia zdobyty przez nich pokarm. Wystarczy aby jedna mr�wka znalaz�a miejsce z pokarmem, to po powrocie do mrowiska inne osobniki s� w stanie odtworzy� drog� do po�ywienia. Na tym etapie nale�a�oby si� zastanowi�, w jaki spos�b mr�wki s� w stanie komunikowa� si� mi�dzy sob�?

Jednym z opisywanych przez nas rozwi�za� do wyznaczania zoptymalizowanej trasy jest algorytm mr�wkowy, inaczej nazywany ACO - Ant Colony Optimization. Pomys� ten zosta� zaczerpni�ty z natury. Jak sama nazwa wskazuje dzia�anie algorytmy jest zwi�zane z mr�wkami, a dok�adnie z koloni� mr�wek. Pomys� na algorytm zosta� zaproponowany na pocz�tku lat 90 XX wieku przez w�oskiego badacza - Marco Dorigo.

Tak jak zosta�o wspomniane wcze�niej, algorytm opiera si� na pracy mr�wek. Chodzi tutaj dok�adnie o tras� jak� mr�wki pokonuj� od swojego siedliska do miejsca w kt�rym znajduje si� po�ywienie. Wa�ne jest znaczenie tutaj pracy zespo�owej. Tras� kszta�tuje ca�a kolonia mr�wek, a nie pojedyncze przypadki. Mr�wki z ka�d� kolejn� podr� wykszta�caj� coraz to bardziej optymaln� tras�.

\section{Opis dzia�ania algorytmu}

Mr�wka w celu znalezienia pokarmu w spos�b losowy wyrusza z mrowiska. Losowo przesuwaj�c si� po terenie szuka pokarmu. Gdy ju� go znajdzie wraca do siedliska i informuje o tym fakcie pozosta�e mr�wki. Aby dostarczy� wi�cej pokarmu mr�wki wyruszaj� do miejsca spoczynku po�ywienia. Chc�c unikn�� sytuacji w kt�rej zdobycz mo�e zosta� zabrana przez inne owady, mr�wki musz� jak najbardziej zoptymalizowa� tras� jak� maj� do pokonania.

Mimo posiadania informacje o znalezionym pokarmie, ka�da mr�wka sama musi zlokalizowa� �r�d�o. Czy mr�wki poruszaj� si� t� sam� tras� przy ka�dym wyj�ciu z mrowiska? Aby trafi� do miejsca w kt�rym znajduje si� pokarm, wspomniane owady wykorzystuj� �lady pozostawione przez osobnik�w kt�re ju� natrafi�y na po�ywienie. W ten spos�b mr�wka kt�ra wyrusza w spos�b losowy, natrafia na �lad poprzednika kt�ry jest wskaz�wk� do znalezienia poszukiwanego pokarmu. Wspomniany �lad nazywa si� feromonem. To dzi�ki tej w�a�ciwo�ci mr�wki s� w stanie lokalizowa� trasy prowadz�ce do pokarmu.

\begin{figure}[h]
	\centering
	\includegraphics[scale=0.75]{grafika/kl_przyklad1}
	\caption{Po�o�enie �cie�ek na przyk�adowym grafie}
	\label{fig:kl_przyklad1}
\end{figure}

Na rysunku \ref{kl_przyklad1} zosta� przedstawiony graf z wierzcho�kami A-B-C-D-E-F. Nad kraw�dziami kolorem czerwonym zosta�y oznaczone wagi. Na pocz�tku za��my, �e mamy do dyspozycji 80 agent�w. Przez N pierwszych iteracji mr�wki porusza�y si� losowo i powsta� nast�puj�cy podzia�:

\begin{figure}[h]
	\centering
	\includegraphics[scale=0.75]{grafika/kl_przyklad2}
	\caption{Rozk�ad agent�w na grafie po N pocz�tkowych iteracjach}
	\label{fig:kl_przyklad2}
\end{figure}

Tak jak mo�na to zauwa�y� na grafice \ref{kl_przyklad2}, po pierwszych iteracjach, przez obie �cie�ki przechodzi taka sama liczba agent�w. Dzieje si� tak poniewa� mr�wki rozpoczynaj� prac� w spos�b losowy. W dalszych krokach nast�puj� modyfikacje i agenci d��� do wyznaczenia najoptymalniejszej �cie�ki.

\begin{figure}[h]
	\centering
	\includegraphics[scale=0.75]{grafika/kl_przyklad3}
	\caption{Rozk�ad agent�w w grafie po optymalizacji}
	\label{fig:kl_przyklad3}
\end{figure}

Liczba agent�w odwiedzaj�cych �cie�ki zmienia si�. Bardziej optymalna trasa zyskuje widoczn� przewag�. W kolejnych iteracjach mr�wki wykorzystuj� si�� feromon�w. Bardziej optymalne �cie�ki s� cz�ciej odwiedzane w zwi�zku z czym zapach na tych kraw�dziach jest silniejszy oraz podtrzymywany. Na mniej optymalnych trasach zapach zanika i przestaj� one by� atrakcyjne dla agent�w. Opisana sytuacja prowadzi do wyznaczenia trasy kt�ra jest najatrakcyjniejsza do przebycia dla mr�wek.

\section{Feromony}

Feromony posiadaj� cech� kt�ra mo�e si� wydawa�, �e negatywnie wp�ywa na wyznaczanie �cie�ki. Chodzi tutaj o ulatnianie si� zostawionego zapachu. Na pierwszy rzut oka mo�e si� to zjawisko wydawa� niepo��danym, ale w rzeczywisto�ci ma du�y wp�ywa na optymalizacj�. Je�li feromony nie straci�yby na swojej sile, to bardzo prawdopodobne, �e pierwotna �cie�ka mog�aby zosta� uznana za najbardziej optymaln�.

W jaki spos�b wyznaczona zostaje najbardziej optymalna �cie�ka? Zapach feromon�w jest podtrzymywany przez w�druj�ce mr�wki. Z czasem owady te same zbaczaj� z drogi w celu poszukiwania alternatywnej trasy. Je�li wybrana trasa jest optymalniejsza od pozosta�ych to �lad jest podtrzymywany, a na innych zanika. Dzi�ki temu w spos�b iteracyjny mo�na wyr�ni� tras� najoptymalniejsz�, a s�absze z czasem zostaj� odrzucone poniewa� przestaj� by� odwiedzane.

Feromony s� istotnym czynnikiem w ca�ym procesie. To dzi�ki nim trasa jest ulepszana. Zapach posiada jedn� z charakterystyk kt�ra na pocz�tku mo�e wydawa� si� problematyczna. Wraz z up�ywem czasu si�a zapachu s�abnie a� do ca�kowitego zanikni�cia. W�a�ciwo�� ta jest zalet�, a nie wad�. To dzi�ki pracy zespo�owej zapach na najlepszej trasie jest podtrzymywany, a na s�abszych zanika. Dzi�ki tej selekcji d�u�sze trasy nie s� brane pod uwag� w wyniku czego zostaje trasa najkorzystniejsza.

Do wyznaczenia optymalnej trasy potrzebne s� d�ugo�ci jakie nale�y przeby� do przemieszczania si� mi�dzy punktami. W \ref{tabela_kosztow_ant} przedstawione s� przyk�adowe odleg�o�ci.

\begin{table}[htb]
\centering
\caption[Kr�tki podpis tabeli 1 -- do spisu tre�ci]{Warto�ci koszt�w}
\begin{tabular}{|c|c|c|c|c|}
\hline
  & A & B & C & D \\\hline
A & 0 & 3 & 8 & 2 \\\hline
B & 3 & 0 & 2 & 4 \\\hline
C & 8 & 2 & 0 & 1 \\\hline
D & 2 & 4 & 1 & 0 \\\hline
\end{tabular}
\label{tabela_kosztow_ant}
\end{table}

\begin{figure}[h]
	\centering
	\includegraphics[scale=0.75]{grafika/kl_init_trasa_alg.png}
	\caption{Graf z wagami}
	\label{fig:kl_init_trasa_alg}
\end{figure}

R�wnie wa�ne jest wyznaczenie pocz�tkowych wsp�czynnik�w feromon�w. Na pocz�tku nadajmy wszystkim kraw�dziom w grafie warto�ci r�wne 1, a wsp�czynnik parowania niech wynosi 0.5.

\begin{table}[htb]
\centering
\caption[Kr�tki podpis tabeli 1 -- do spisu tre�ci]{Warto�ci feromon�w}
\begin{tabular}{|c|c|c|c|c|}
\hline
  & A & B & C & D \\\hline
A & 0 & 1 & 1 & 1 \\\hline
B & 1 & 0 & 1 & 1 \\\hline
C & 1 & 1 & 0 & 1 \\\hline
D & 1 & 1 & 1 & 0 \\\hline
\end{tabular}
\label{tabela_feromonow_ant}
\end{table}

\begin{figure}[h]
	\centering
	\includegraphics[scale=0.75]{grafika/kl_init_p_alg.png}
	\caption{Graf z pocz�tkowymi warto�ciami feromon�w}
	\label{fig:kl_init_trasa_alg}
\end{figure}

Posiadaj�c pocz�tkowe dane wyznaczmy w spos�b losowy trasy dla dw�ch agent�w: L1 i L2. Agent L1 porusza� si� tras� w kt�rej odwiedzi� wierzcho�ki w nast�puj�cej kolejno�ci: A, B, C, D, A. 

\begin{figure}[h]
	\centering
	\includegraphics[scale=0.75]{grafika/kl_l1_trasa_alg.png}
	\caption{Trasa przebyta przez agenta L1}
	\label{fig:kl_init_trasa_alg}
\end{figure}

Agent L2 wyznaczy� nast�puj�c� tras�: A, C, B, D, A. 

\begin{figure}[h]
	\centering
	\includegraphics[scale=0.75]{grafika/kl_l2_trasa_alg.png}
	\caption{Trasa przebyta przez agenta L2}
	\label{fig:kl_init_trasa_alg}
\end{figure}

Mr�wki odpowiednio pokona�y dystans 8 i 16 punkt�w. Dzi�ki tej informacji mo�na zaktualizowa� warto�ci feromon�w na poszczeg�lnych kraw�dziach. Do oblicze� wykorzystany jest wz�r: $\tau\textsubscript{i,j}=$\((1-p)\tau\textsubscript{i,j}+\sum\limits_{k=1}^m \Delta\tau_{i,j}^k\). Kraw�dzie A-B i C-D zosta�y odwiedzona jedynie przez agenta L1 w wyniku czego warto�� feromonu zostaje zmieniona na 10/16. Nast�pnie kraw�dzie A-C i B-D zostaj� zaktualizowane na 9/16. Kraw�dzie A-D i B-C s� odwiedzone dwukrotnie a warto�� feromon�w wynosi 11/16.

\begin{figure}[h]
	\centering
	\includegraphics[scale=0.75]{grafika/kl_p_alg.png}
	\caption{Graf z warto�ciami feromon�w po modyfikacji}
	\label{fig:kl_init_trasa_alg}
\end{figure}

Ostatni� faz� algorytmu jest wyznaczenie prawdopodobie�stwa z jakim kolejni agenci b�d� wybiera� kolejny wierzcho�ek. W kolejnej iteracji agent L3 znajduje si� w wierzcho�ku B. Do wyboru ma kraw�dzie A, C i D. Wed�ug wzoru $p\textsubscript{i,j}=\frac{(\tau\textsubscript{i,j})^\alpha(\eta\textsubscript{i,j})^\beta}{\sum(\tau\textsubscript{i,j})^\alpha(\eta\textsubscript{i,j})^\beta)}\) wyliczane jest prawdopodobie�stwo dla wszystkich mo�liwo�ci. Przej�cie z kraw�dzi B do kraw�dzi A wynosi oko�o 20\%, do kraw�dzi D 30\%, a do kraw�dzi C oko�o 50\%. 

Optymalna trasa zostanie wykszta�cona po wykonaniu wielu iteracji. Ilo�� iteracji nie jest zdefiniowana i dla ka�dego przypadku mo�e by� r�na. Sytuacja wygl�da identycznie w przypadku wyboru warto�ci p. Wa�ne jest natomiast to, aby w trakcie dzia�ania algorytmu nie modyfikowa� tej warto�ci. Powinno ona by� taka sama na ka�dym kroku algorytmu.


\chapter{Algorytmy zach�anne - Przemys�aw}
	Istnieje wiele algorytm�w zach�annych pozwalaj�cych otrzyma� optymalne rozwi�zanie dla problemu znalezienia najkr�tszej trasy. W poni�szym rozdziale opisane dok�adniej zosta�o dzia�anie algorytmu najbli�szego s�siada, algorytmu najmniejszej kraw�dzi oraz algorytmu A*. W celu zoptymalizowania otrzymanych rozwi�za� zastosowany zostanie operator 2-opt oraz wsp�czynnik b��du przy dokonywaniu wybor�w.

\section{Algorytm najbli�szego s�siada}
	Poni�szy podrozdzia� przybli�y dzia�anie algorytmu najbli�szego s�siada. Wszystkie wymagane dla algorytmu operacje zostan� opisane krok po kroku oraz przedstawione na kr�tkim przyk�adzie. Jak sama nazwa wskazuje jest to algorytm polegaj�cy na odwiedzaniu, zaczynaj�c od wybranego wierzcho�ka pocz�tkowego nast�pnego wierzcho�ka jeszcze nieodwiedzonego znajduj�cego si� najbli�ej poprzednio odwiedzonego. 

\begin{figure}[h]
	\centering
	\includegraphics[scale=0.5]{grafika/ans_graph.png}
	\caption{Schemat algorytmu najbli�szego s�siada}
	\label{nearest_neighbour}
\end{figure}

Na rysunku \ref{nearest_neighbour} zosta� pokazany schemat blokowy algorytmu najbli�szego s�siada. Pierwszym krokiem jest wyznaczenie wyznaczenie wierzcho�ka startowego. Nast�pnie dla aktualnego wierzcho�ka nale�y obliczy� najkr�tsz� kraw�d� ��cz�c� aktualny wierzcho�ek spo�r�d kolekcji wierzcho�k�w nieodwiedzonych. Wyb�r najlepszej opcji po��czenia dw�ch wierzcho�k�w nale�y doda� do rozwi�zania, a drugi wierzcho�ek staje si� aktualnym od kt�rego b�dziemy wyznacza� kolejne odleg�o�ci do nieodwiedzonych jeszcze wierzcho�k�w. Czynno�ci nale�y powtarza� do momentu odwiedzenia wszystkich wierzcho�k�w w podanym grafie. Na samym ko�cu wystarczy jedynie po��czy� ostatni wierzcho�ek z pocz�tkowym. Po wykonaniu wszystkich krok�w algorytm najbli�szego s�siada powinien zwr�ci� optymalne dla niego rozwi�zanie.

\begin{figure}[h]
	\centering
	\includegraphics[scale=0.3]{grafika/ans_last.png}
	\caption{Algorytm najbli�szego s�siada}
	\label{nearest_neighbour_inaction}
\end{figure}

	Aby lepiej zobrazowa� otrzymanie rozwi�zania przez algorytm przedstawiono poni�ej na rysunku \ref{nearest_neighbour_inaction} jego dzia�anie na przyk�adzie. W podanym przypadku nale�y za�o�y� �e wierzcho�ek nr 2 jest wierzcho�kiem startowym, wi�c nale�y ustawi� go jako aktualny w danym momencie. Po obliczeniu wszystkich odleg�o�ci prowadzonych do wierzcho�k�w nieodwiedzonych wychodzi, �e najkr�tsza kraw�d� prowadzi do wierzcho�ka nr 4, wi�c nale�y do��czy� wybrana kraw�d� do rozwi�zania i oznaczy� nowy aktualny wierzcho�ek, kt�ry staje si� r�wnie� odwiedzonym. W podanym grafie znajduj� si� nadal nieodwiedzone wierzcho�ki, dlatego algorytm powtarza krok nr 2 w celu znalezienia kolejnej najkr�tszej kraw�dzi. Kolejn� najlepsz� odnalezion� w danym momencie kraw�dzi�, kt�r� algorytm doda do rozwi�zania b�dzie kraw�d� o warto�ci 3 ��cz�ca aktualny wierzcho�ek z wierzcho�kiem nr 5. Kroki 4, 5 oraz 6 przedstawiaj� kolejne powtarzalne iteracje algorytmu dochodz�c w ostatnim kroku do utworzenia cyklu i tym zako�czeniu dzia�ania algorytmu. W ten spos�b otrzymano zach�anne rozwi�zanie 2 - 4 - 5 - 3 - 1 - 2. Warto zaznaczy�, �e dzi�ki oznaczaniu wierzcho�k�w jako odwiedzone nie trzeba w �adnej iteracji martwi� si� o to czy do��czenie kolejnej kraw�dzi z nieodwiedzonym wierzcho�kiem spowoduje utworzenie niepo��danego cyklu.

\section{Algorytm najmniejszej kraw�dzi}
	Kolejny podrozdzia� algorytm�w zach�annych zosta� po�wi�cony dok�adniejszemu opisie dzia�ania algorytmu najmniejszej kraw�dzi, kt�ry w swoim dzia�aniu przypomina algorytm poszukuj�cy minimalnego drzewa rozpinaj�cego, poprzez do��czenie do aktualnego rozwi�zania najkr�tszych w�r�d dopuszczalnych kraw�dzi.	Aby otrzyma� zach�anne rozwi�zanie przy wykorzystaniu algorytmu najmniejszej kraw�dzi nale�y wykona� nast�puj�ce kroki przedstawione schemacie blokowym na rysunku \ref{ank}.
	
\begin{figure}[h]
	\centering
	\includegraphics[scale=0.4]{grafika/ank_graph.png}
	\caption{Schemat algorytmu najmniejszej kraw�dzi}
	\label{ank}
\end{figure}

Algorytm rozpoczyna swoje dzia�anie od posortowania wag kraw�dzi mi�dzy wszystkimi wierzcho�kami oraz umieszcza podane odleg�o�ci w kolekcji. Nast�pnie wybierana jest zawsze kraw�d� o najmniejsze warto�ci oraz od razu usuwana jest z podanej kolekcji. Aby wybrane po��czenie mi�dzy wierzcho�kami zosta�o dodane do rozwi�zania musi spe�nia� warunek nie utworzenia cyklu oraz wierzcho�ka o trzech kraw�dziach, w przeciwnym wypadku po��czenie jest pomijane oraz algorytm wybiera kolejn� kraw�d�. Iteracje s� powtarzane do momentu, a� liczba dodanych po��cze� jest r�wna liczbie wszystkich wierzcho�k�w. W przypadku ostatniej iteracji wyznaczona kraw�d� nie musi ju� spe�nia� warunku z utworzeniem cyklu. Po zako�czeniu dzia�ania algorytmu otrzymano optymalne rozwi�zanie dla algorytmu najmniejszej kraw�dzi.
	
\begin{figure}[h]
	\centering
	\includegraphics[scale=0.5]{grafika/ank_initial_finish.png}
	\caption{Pocz�tkowe rozmieszczenie wierzcho�k�w}
	\label{ank_initial}
\end{figure}

Dzia�anie algorytmu przedstawione zosta�o na rysunku \ref{ank_initial} przedstawiaj�cym przyk�ad ze sze�cioma losowo rozmieszczonymi wierzcho�kami. Natomiast wizualizacja kolejnych krok�w algorytmu zosta�a przedstawiona na rysunku \ref{ank_final}. Po posortowaniu wszystkich dost�pnych kraw�dzi dla ka�dego wierzcho�ka mo�na rozpocz�� dzia�anie algorytmu.
	
\begin{figure}[h]
	\centering
	\includegraphics[scale=0.30]{grafika/ank_steps_final.png}
	\caption{Algorytm najmniejszej kraw�dzi}
	\label{ank_final}
\end{figure}

	Podczas pierwszej iteracji okazuje si�, �e s� aktualnie dwie kraw�dzie z najmniejsz� wag� o warto�ci 3 (1 - 2 oraz 4 - 5), nie ma wi�c znaczenia, kt�ra kraw�d� algorytm doda w pierwszej kolejno�ci, poniewa� �adna z wybranych nie utworzy w tym momencie cyklu. W przypadku drugiej iteracji algorytm wybiera kraw�d� o najmniejszej wadze, do��czaj�c j� do aktualnego rozwi�zania. Analogiczna sytuacja do pierwszej iteracji wyst�puje w trzeciej iteracji, gdzie nie ma r�nicy, kt�ra kraw�d� zostanie do��czona do rozwi�zania. Czwarta iteracja pokazuje sytuacje, w kt�rej nie mo�na po��czy� kraw�dzi 3 - 4, poniewa� utworzy�oby to niedozwolony w trakcie algorytmu cykl, dlatego tym razem wybierane jest inne po��czenie o najmniejszej wadze (4 - 6). Kolejny krok przedstawia przedstawia przypadek, gdzie nie jest mo�liwe po��czenie kraw�dzi 2 - 4 (najmniejsza warto�� - 7), poniewa� spowodowa�oby dla wierzcho�ka nr 4 utworzenie trzech wychodz�cych z niego kraw�dzi, wi�c do rozwi�zania dochodzi kolejna najmniejsza kraw�d� 1 - 6. W ostatnim kroku, pomimo dost�pnych kraw�dzi z mniejsz� wag� wybierana zosta�a kraw�d� 2 - 3, poniewa� tylko ona nie spowoduje utworzenia wierzcho�ka o trzech kraw�dziach. W takim wypadku ko�cz�c dzia�anie algorytmu otrzymano rozwi�zanie 1 - 6 - 4 - 5 - 3 - 2 - 1.
	
\section{Algorytm A*}
	Ostatnim om�wionym rozwi�zaniem do znajdowania najkr�tszej �cie�ki w grafie jest algorytm A*, w kt�rym zawsze zostanie znalezione najkorzystniejsze zach�anne rozwi�zanie. Strategia gwarantuje, �e ka�dy wierzcho�ek zostanie odwiedzony, przy czym dokonuje w danym momencie najlepszych wybor�w. G��wn� zasad� algorytmu A* jest minimalizacja funkcji kosztu $g(x)$ oraz funkcji heurystycznej $h(x)$ zdefiniowanej jako funkcja celu $f(x)=g(x)+h(x)$. Funkcja heurystyczna musi spe�nia� dwa wymagane warunki tj. warunek dopuszczalno�ci i warunek monotoniczno�ci. Warunek dopuszczalno�ci polega na tym, aby funkcja heurystyczna za bardzo minimalizowa�a koszt, a przesadza�a przy maksymalizacji zysku. Natomiast warunek monotoniczno�ci m�wi, �e oszacowywanie wyniku musi by� coraz mniej optymistyczne w momencie zbli�ania si� do rozwi�zania. Je�li przestrze� przeszukiwa� zawiera� b�dzie �cie�ki to mo�na w�wczas sprowadzi� problem do problemu poszukiwania najkr�tszej �cie�ki w grafie. W danym przypadku funkcj� heurystyczn� b�dzie w�wczas odleg�o�� w linii prostej mi�dzy wierzcho�kami. Zapewnia to w ten spos�b, �e taka funkcja jest nadmiernie optymistyczna.
	
\begin{figure}[h]
	\centering
	\includegraphics[scale=0.4]{grafika/astar_schema.png}
	\caption{Schemat blokowy algorytmu A*}
	\label{astar_schema}
\end{figure}	
	
Schemat blokowy dzia�ania algorytmu A* zosta� przedstawiony na rysunku \ref{astar_schema}. Algorytm rozpoczyna swoje dzia�anie od wyboru wierzcho�ka startowego oraz ko�cowego, gdzie wierzcho�ek startowy staje si� aktualnym. Nast�pnie dla aktualnego wierzcho�ka obliczana i zapisywana jest funkcja celu $f(x)$ dla ka�dego poszczeg�lnego mo�liwego po��czenia. Kolejnym krokiem jest wybranie spo�r�d minimalnej funkcji celu dla aktualnego wierzcho�ka oraz minimalnej funkcji celu z poprzednio zapisanych (je�li takowe istniej�). Je�li wybrana funkcja celu nie ��czy aktualnego wierzcho�ka z docelowym oraz nie utworzy cyklu lub wierzcho�ka o trzech kraw�dziach jest brana dalej pod uwag� i ustalany jest nowy aktualny wierzcho�ek. W przeciwnym wypadku wybrana funkcja celu nie jest brana pod uwag� i algorytm jeszcze raz por�wnuje obliczone poprzednio funkcje celu. Kroki algorytmu s� powtarzane do momentu a� wybrana funkcja celu po��czy aktualny wierzcho�ek z ko�cowym.


\begin{figure}[h]
	\centering
	\includegraphics[scale=0.4]{grafika/astar_initial.png}
	\caption{Graf u�yty do przedstawienia dzia�ania algorytmu A*}
	\label{astar_initial}
\end{figure}

Lepsza wizualizacja dzia�ania algorytmu A* zosta�a przedstawiona na grafie umieszczonym na rysunku \ref{astar_initial}. Wierzcho�kiem startowym b�dzie w tym wypadku wierzcho�ek numer 1, natomiast wierzcho�kiem docelowym b�dzie punkt numer 7. Jak mo�na zauwa�yc na rysunku \ref{astar_final} podczas pierwszej iteracji algorytm A* zdefiniowa� dwie funkcje celu, jednak�e $f1(2)=4+5$ jest w danym momencie lepszym wyborem. W kolejnej iteracji nadal rozwini�cie pierwszej funkcji celu daje lepszy rezultat, dlatego te� nowym aktualnym wierzcho�kiem staje si� wierzcho�ek numer 3. Trzecia iteracja przedstawia sytuacje, w kt�rej druga funkcja celu ��cz�ca aktualny wierzcho�ek z wierzcho�kiem numer 5 daje lepszy w danym momencie rezultat, dlatego wierzcho�ek numer 5 jest brany pod uwag�. Po zastosowaniu si� do za�o�e� algorytmu oraz wykonaniu wszystkich krok�w otrzymujemy optymalne rozwi�zanie ��cz�ce wierzcho�ek numer 1 z wierzcho�kiem numer 7(1 - 5 - 6 - 7) o warto�ci 21.


\begin{figure}[h]
	\centering
	\includegraphics[scale=0.2]{grafika/astar_final.png}
	\caption{Poszczeg�lne iteracje algorytmu A*}
	\label{astar_final}
\end{figure}



\section{Optymalizacja otrzymanych rozwi�za�}
	W ostatnim podrozdziale skupimy si� na sposobach optymalizacji otrzymanych przez algorytmy zach�anne rozwi�za�. Dok�adniej om�wiona zostanie metoda 2-opt oraz zastosowanie wsp�czynnika b��du zaproponowanego przez studenta.
	
\subsection{Metoda 2-opt}
	Metoda optymalizacyjna 2-opt polega na pozbyciu si� z cyklu dw�ch kraw�dzi w celu zast�pienia ich innymi kraw�dziami w taki spos�b, aby otworzy� zupe�nie inny cykl. Iteracje mo�na powtarza� dla ka�dej pary kraw�dzi, opr�cz tych s�siaduj�cych ze sob�, poniewa� ich zamienienie nie przynios�oby �adnej modyfikacji. Metoda nie ma na celu zmiany po�o�enia wierzcho�k�w, jedynie kolejno�ci ich odwiedzania. Po wykonaniu ca�ej optymalizacji, nale�y sprawdzi� kt�ra modyfikacja przynios�a najlepszy efekt skr�cenia d�ugo�ci cyklu. W przypadku je�eli �adna modyfikacja nie da�a lepszego rozwi�zania, nie nale�y modyfikowa� rozwi�zania. Algorytm mo�na wykonywa� wielokrotnie, w ten spos�b zostanie zrealizowane minimum lokalne. Dla lepszego przedstawienia dzia�ania metody 2-opt, nale�y spojrze� na rysunek 5.4.

\begin{figure}[h]
	\centering
	\includegraphics[scale=0.7]{grafika/2opt.png}
	\caption{Metoda 2-opt}
	\label{2opt}
\end{figure}	

\subsection{Propozycje optymalizacji}
	Aby rozwa�y� wi�cej tras zwracanych przez algorytmy zach�anne zaproponowano u�ycie wsp�czynnika b��du przy dokonywaniu przez algorytm w danym momencie najlepszego dla niego wyboru. Wsp�czynnik dzia�a na zasadzie brania pod uwag� gorszego wyboru, aby zwi�kszy� liczb� przegl�danych odcink�w. Ma to na celu mo�liwo�� znalezienia lepszego rozwi�zania ni� takiego jakie zwr�ci�by sam algorytm. Wsp�czynnik b��du mo�e by� uruchamiany na ka�dym etapie budowania docelowej trasy. Zastosowanie wsp�czynnika b��du mo�e podnie�� z�o�ono�� algorytmu nawet do $\theta$(\(n^n\)), gdzie $n$ jest r�wne liczbie wierzcho�ku. St�d propozycj� jest stosowanie wsp�czynnika b��du tylko na wybranych trzech etapach budowania ko�cowej trasy.

\chapter{Badania}

\section{Wyniki algorytmu genetycznego}

\section{Wyniki algorytmu mr�wkowego}

\section{Wyniki algorytmu mr�wkowego}

\chapter{Badania}

\section{Trasa I}

\subsection{Wyniki algorytmu genetycznego}

\begin{table}[]
\caption[Trasa I - �rednie wyniki dla algorytmu genetycznego]
{Trasa I - �rednie wyniki algorytmu genetycznego dla danych parametr�w wej�ciowych}
\scalebox{0.8}{
\begin{tabular}{|l|l|l|l|l|l|l|l|l|l|l|}
\hline
\rowcolor[HTML]{C0C0C0} 
\multicolumn{1}{|c|}{\cellcolor[HTML]{C0C0C0}}                    & \multicolumn{1}{c|}{\cellcolor[HTML]{C0C0C0}}                    & PMX                           & PMX                           & PMX                           & OX                            & OX                            & OX                            & CX                            & CX                            & CX                            \\ \cline{3-11} 
\rowcolor[HTML]{C0C0C0} 
\multicolumn{1}{|c|}{\multirow{-2}{*}{\cellcolor[HTML]{C0C0C0}N}} & \multicolumn{1}{c|}{\multirow{-2}{*}{\cellcolor[HTML]{C0C0C0}k}} & Insert                        & Replace                       & Revert                        & Insert                        & Replace                       & Revert                        & Insert                        & Replace                       & Revert                        \\ \hline
\rowcolor[HTML]{FFFFFF} 
\cellcolor[HTML]{C0C0C0}100                                       & \cellcolor[HTML]{C0C0C0}100                                      & {\color[HTML]{333333} 91502}  & 96029                         & 105868                        & 102746                        & 109380                        & 115816                        & 97015                         & 112795                        & 116291                        \\ \hline
\rowcolor[HTML]{FFFFFF} 
\cellcolor[HTML]{C0C0C0}100                                       & \cellcolor[HTML]{C0C0C0}1000                                     & 52959                         & 67241                         & 80988                         & 58800                         & 70722                         & 93124                         & 55599                         & 67370                         & 92538                         \\ \hline
\rowcolor[HTML]{FFFFFF} 
\cellcolor[HTML]{C0C0C0}100                                       & \cellcolor[HTML]{C0C0C0}2000                                     & 49747                         & 61007                         & 80468                         & 52822                         & 64938                         & 89324                         & 51134                         & 61108                         & 86920                         \\ \hline
\rowcolor[HTML]{FFFFFF} 
\cellcolor[HTML]{C0C0C0}100                                       & \cellcolor[HTML]{C0C0C0}4000                                     & 49368                         & 59649                         & 79494                         & 50284                         & 62967                         & 98000                         & \cellcolor[HTML]{FFCB2F}48571 & 57028                         & 85739                         \\ \hline
\rowcolor[HTML]{FFFFFF} 
\cellcolor[HTML]{C0C0C0}1000                                      & \cellcolor[HTML]{C0C0C0}100                                      & 55008                         & 67898                         & 56269                         & 64427                         & 63525                         & 69052                         & {\color[HTML]{333333} 49959}  & 53471                         & 58104                         \\ \hline
\rowcolor[HTML]{FFFC9E} 
\cellcolor[HTML]{C0C0C0}1000                                      & \cellcolor[HTML]{C0C0C0}1000                                     & 37347                         & 42081                         & \cellcolor[HTML]{FFFFFF}49527 & 41809                         & 46600                         & \cellcolor[HTML]{FFFFFF}60650 & \cellcolor[HTML]{FFCB2F}33768 & 37948                         & 50090                         \\ \hline
\rowcolor[HTML]{FFFC9E} 
\cellcolor[HTML]{C0C0C0}1000                                      & \cellcolor[HTML]{C0C0C0}2000                                     & 39158                         & 40581                         & \cellcolor[HTML]{FFFFFF}50985 & 40903                         & 45041                         & \cellcolor[HTML]{FFFFFF}64489 & \cellcolor[HTML]{FFCB2F}33843 & 38791                         & 48072                         \\ \hline
\cellcolor[HTML]{C0C0C0}1000                                      & \cellcolor[HTML]{C0C0C0}4000                                     & \cellcolor[HTML]{FFFC9E}37951 & \cellcolor[HTML]{FFFC9E}42112 & \cellcolor[HTML]{FFFFFF}50359 & \cellcolor[HTML]{FFFC9E}40036 & \cellcolor[HTML]{FFFC9E}47795 & \cellcolor[HTML]{FFFFFF}62245 & \cellcolor[HTML]{FFCB2F}34961 & \cellcolor[HTML]{FFFC9E}37842 & \cellcolor[HTML]{FFFFFF}49857 \\ \hline
\rowcolor[HTML]{FFFFFF} 
\cellcolor[HTML]{C0C0C0}2000                                      & \cellcolor[HTML]{C0C0C0}100                                      & 51624                         & 48902                         & 49112                         & 62336                         & 61150                         & 57542                         & \cellcolor[HTML]{FFCB2F}40969 & \cellcolor[HTML]{FFFC9E}44643 & \cellcolor[HTML]{FFFC9E}42287 \\ \hline
\rowcolor[HTML]{FFFC9E} 
\cellcolor[HTML]{C0C0C0}2000                                      & \cellcolor[HTML]{C0C0C0}1000                                     & 36492                         & 39233                         & 43081                         & 38086                         & 45761                         & \cellcolor[HTML]{FFFFFF}53166 & \cellcolor[HTML]{FFCB2F}32571 & 35453                         & 39076                         \\ \hline
\rowcolor[HTML]{FFFC9E} 
\cellcolor[HTML]{C0C0C0}2000                                      & \cellcolor[HTML]{C0C0C0}2000                                     & 36410                         & 40379                         & 40711                         & 39554                         & 41320                         & \cellcolor[HTML]{FFFFFF}52030 & \cellcolor[HTML]{34FF34}32326 & 35353                         & 40382                         \\ \hline
\rowcolor[HTML]{FFFC9E} 
\cellcolor[HTML]{C0C0C0}2000                                      & \cellcolor[HTML]{C0C0C0}4000                                     & 35406                         & 38159                         & 42454                         & 39145                         & 44137                         & \cellcolor[HTML]{FFFFFF}51350 & \cellcolor[HTML]{FFCB2F}32667 & 34999                         & 48295                         \\ \hline
\end{tabular}}
\end{table}

\begin{table}[]
\caption[Trasa I - najlepsze wyniki dla algorytmu genetycznego]
{Trasa I - najlepsze wyniki algorytmu genetycznego dla danych parametr�w wej�ciowych}
\scalebox{0.8}{
\begin{tabular}{|l|l|l|l|l|l|l|l|l|l|l|}
\hline
\rowcolor[HTML]{C0C0C0} 
\multicolumn{1}{|c|}{\cellcolor[HTML]{C0C0C0}}                    & \multicolumn{1}{c|}{\cellcolor[HTML]{C0C0C0}}                    & PMX                           & PMX                           & PMX                           & OX                            & OX                            & OX                            & CX                                                   & CX                            & CX                            \\ \cline{3-11} 
\rowcolor[HTML]{C0C0C0} 
\multicolumn{1}{|c|}{\multirow{-2}{*}{\cellcolor[HTML]{C0C0C0}N}} & \multicolumn{1}{c|}{\multirow{-2}{*}{\cellcolor[HTML]{C0C0C0}k}} & Insert                        & Swap                          & Revert                        & Insert                        & Swap                          & Revert                        & Insert                                               & Swap                          & Revert                        \\ \hline
\rowcolor[HTML]{FFFFFF} 
\cellcolor[HTML]{C0C0C0}100                                       & \cellcolor[HTML]{C0C0C0}100                                      & 80385                         & 87226                         & 81154                         & 90764                         & 99185                         & 101137                        & 72268                                                & 103408                        & 92859                         \\ \hline
\rowcolor[HTML]{FFFFFF} 
\cellcolor[HTML]{C0C0C0}100                                       & \cellcolor[HTML]{C0C0C0}1000                                     & \cellcolor[HTML]{FFCB2F}45076 & 62481                         & 67443                         & 53948                         & 60699                         & 78382                         & \cellcolor[HTML]{FFFC9E}45134                        & 56918                         & 79955                         \\ \hline
\rowcolor[HTML]{FFFFFF} 
\cellcolor[HTML]{C0C0C0}100                                       & \cellcolor[HTML]{C0C0C0}2000                                     & \cellcolor[HTML]{FFCB2F}40964 & 53316                         & 70100                         & 45500                         & 55396                         & 80719                         & \cellcolor[HTML]{FFFC9E}43142                        & 54430                         & 79636                         \\ \hline
\rowcolor[HTML]{FFFFFF} 
\cellcolor[HTML]{C0C0C0}100                                       & \cellcolor[HTML]{C0C0C0}4000                                     & \cellcolor[HTML]{FFFC9E}44853 & 52397                         & 69287                         & 40628                         & 56153                         & 91440                         & \cellcolor[HTML]{FFCB2F}41805                        & \cellcolor[HTML]{FFFC9E}45953 & 72035                         \\ \hline
\cellcolor[HTML]{C0C0C0}1000                                      & \cellcolor[HTML]{C0C0C0}100                                      & \cellcolor[HTML]{FFFC9E}45769 & \cellcolor[HTML]{FFFFFF}59523 & \cellcolor[HTML]{FFFFFF}50232 & \cellcolor[HTML]{FFFFFF}56634 & \cellcolor[HTML]{FFFFFF}58272 & \cellcolor[HTML]{FFFFFF}62965 & \cellcolor[HTML]{FFCB2F}{\color[HTML]{333333} 40804} & \cellcolor[HTML]{FFFC9E}42391 & \cellcolor[HTML]{FFFC9E}41647 \\ \hline
\rowcolor[HTML]{FFFC9E} 
\cellcolor[HTML]{C0C0C0}1000                                      & \cellcolor[HTML]{C0C0C0}1000                                     & 34630                         & 36034                         & 42016                         & 39283                         & 44320                         & \cellcolor[HTML]{FFFFFF}53480 & \cellcolor[HTML]{FFCC67}32433                        & 35227                         & 37569                         \\ \hline
\rowcolor[HTML]{FFFC9E} 
\cellcolor[HTML]{C0C0C0}1000                                      & \cellcolor[HTML]{C0C0C0}2000                                     & 35051                         & 36704                         & 45852                         & 35709                         & 41317                         & \cellcolor[HTML]{FFFFFF}57478 & \cellcolor[HTML]{FFCB2F}32334                        & 35576                         & 39953                         \\ \hline
\rowcolor[HTML]{FFFC9E} 
\cellcolor[HTML]{C0C0C0}1000                                      & \cellcolor[HTML]{C0C0C0}4000                                     & 33268                         & 37649                         & 44333                         & 35718                         & 41883                         & 49322                         & \cellcolor[HTML]{FFCB2F}33089                        & 34642                         & 40220                         \\ \hline
\rowcolor[HTML]{FFFC9E} 
\cellcolor[HTML]{C0C0C0}2000                                      & \cellcolor[HTML]{C0C0C0}100                                      & 44493                         & 42497                         & 42138                         & 56466                         & 56884                         & \cellcolor[HTML]{FFFFFF}52010 & \cellcolor[HTML]{FFCB2F}36349                        & 38830                         & 38461                         \\ \hline
\rowcolor[HTML]{FFFC9E} 
\cellcolor[HTML]{C0C0C0}2000                                      & \cellcolor[HTML]{C0C0C0}1000                                     & 33609                         & 36100                         & 37047                         & 34801                         & 38625                         & 40817                         & \cellcolor[HTML]{34FF34}30939                        & 33284                         & 36203                         \\ \hline
\rowcolor[HTML]{FFFC9E} 
\cellcolor[HTML]{C0C0C0}2000                                      & \cellcolor[HTML]{C0C0C0}2000                                     & 33257                         & 36216                         & 37967                         & 34591                         & 36458                         & 47022                         & \cellcolor[HTML]{FFCB2F}31118                        & 32798                         & 36203                         \\ \hline
\rowcolor[HTML]{FFFC9E} 
\cellcolor[HTML]{C0C0C0}2000                                      & \cellcolor[HTML]{C0C0C0}4000                                     & 31347                         & 36834                         & 40193                         & 34269                         & 37693                         & 43242                         & \cellcolor[HTML]{FFCB2F}31118                        & 33245                         & 35271                         \\ \hline
\end{tabular}}
\end{table}

\begin{figure}[h]
	\centering
	\includegraphics[scale=0.4]{grafika/Trasa1GA.png}
	\caption{Trasa I - Zale�no�� �redniej odleg�o�ci od czasu}
	\label{mutacja}
\end{figure}	

\subsection{Wyniki algorytmu mr�wkowego}

Jednym za parametr�w wej�ciowych algorytmu mr�wkowego jest wsp�czynnik parowania feromon�w. To dzi�ki temu parametrowi wybierane s� optymalne trasy, a te gorsze odrzucane. 
Ich warto�ci s� wyra�ane procentowo w zakresie od 1\% do 99\%. Dla ka�dej trasy zosta�a wybrane takie same warto�ci. Pomiar wynik�w zosta� rozpocz�ty od warto�ci 10\%. W ka�dym korku warto�� ta by�a zwi�kszana o kolejne 10\% a� do 90\%.

Kolejnym istotnym parametrem jest ilo�� iteracji. Parametr ten oznacza ile razy mr�wki mr�wki maj� wykonywa� ulepszenie trasy. Do bada� zosta�y dobrane trzy warto�ci: 50, 75, 100. Parametr ten ma bezpo�redni wp�yw na optymalizacj� ale r�wnie� na czas jaki algorytm potrzebuje na wykonanie wszystkich operacji. 
	
Algorytm zosta� uruchomiony dziesi�ciokrotne dla ka�dej mo�liwej wariacji parametr�w wej�ciowych. W odpowiedzi program przedstawi� w odpowiedniej kolejno�ci punkty jakie zosta�y zawarte w rozwi�zaniu, przebyty dystans oraz czas wykonania algorytmu. Z zebranych danych zosta�a wyliczona �rednia oraz zosta� wyznaczony najlepszy wynik. Przy wykorzystaniu tych miar mo�na wykona� por�wnanie z oryginaln� tras� przebyt� w rzeczywisto�ci przez ci�ar�wk�. 
	
Ka�da z tabel przedstawiaj�ca wynik algorytmu mr�wkowego sk�ada si� z takiej samej ilo�ci kolumn oraz wierszy. W kolumnie z oznaczeniem \textit{k} zosta�a zaprezentowana ilo�� przeprowadzonych iteracji. Warto�ci te zaczynaj� si� od 50 i s� zwi�kszane co 25 do 100. Ka�da kolumna przedstawia wsp�czynnik parowania feromon�w. Warto�ci te zaczynaj� si� od 0.1 i s� zwi�kszane co 0.1 do 0.9. T�o kom�rek z warto�ciami przebytego dystansu zosta�y zmienione w zale�no�ci od spe�nianego warunku. Kolorem jasno-��tym s� oznaczone warto�ci kt�re s� wi�ksze od oryginalnej trasy. Kolorem ��tym zosta�y oznaczone pole kt�re s� najlepsze dla danego wsp�czynnika \textit{k}. Najlepszy wynik zosta� oznaczony kolorem zielonym. W tabeli sumuj�cej 
		
Najlepsze wyniki algorytmu mr�wkowego dla trasy numer 1 zosta�y przedstawione w \ref{tab:best_1_mrowkowy}. Spo�r�d wszystkich wariacji jedynie dwa wyniki zosta�y sklasyfikowane jako gorsze od oryginalnej odleg�o�ci. Te najgorsze wyniki wyst�pi�y dla najni�szej warto�ci iteracji. Najlepsza warto�� jak� uda�o si� osi�gn�� to 38111. Jest to warto�� jaka zosta�a osi�gni�ta dla 75 ilo�ci iteracji oraz dla wsp�czynnika parowania wynosz�cego 0.2. Warto�ci lepsze od oryginalnych wyst�pi�y dla oko�o 93\% mo�liwych kombinacji parametr�w wej�ciowych.

\begin{table}[h!]
\label{tab:best_1_mrowkowy}
\caption[Trasa I - najlepsze wyniki dla algorytmu mr�wkowego]
{Trasa I - najlepsze wyniki algorytmu mr�wkowego dla danych parametr�w wej�ciowych}
\begin{tabular}{|c|l|l|l|l|l|l|l|l|l|}
\hline
\rowcolor[HTML]{C0C0C0} 
k                           & \multicolumn{1}{c|}{\cellcolor[HTML]{C0C0C0}0.1}     & \multicolumn{1}{c|}{\cellcolor[HTML]{C0C0C0}0.2} & \multicolumn{1}{c|}{\cellcolor[HTML]{C0C0C0}0.3} & \multicolumn{1}{c|}{\cellcolor[HTML]{C0C0C0}0.4} & \multicolumn{1}{c|}{\cellcolor[HTML]{C0C0C0}0.5} & \multicolumn{1}{c|}{\cellcolor[HTML]{C0C0C0}0.6} & \multicolumn{1}{c|}{\cellcolor[HTML]{C0C0C0}0.7} & \multicolumn{1}{c|}{\cellcolor[HTML]{C0C0C0}0.8} & \multicolumn{1}{c|}{\cellcolor[HTML]{C0C0C0}0.9} \\ \hline
\rowcolor[HTML]{FFFC9E} 
\cellcolor[HTML]{C0C0C0}50  & \cellcolor[HTML]{FFFFFF}{\color[HTML]{333333} 57946} & 40150                                            & 39964                                            & \cellcolor[HTML]{FFCB2F}38773                    & 47140                                            & 45598                                            & \cellcolor[HTML]{FFFFFF}57402                    & 45451                                            & \cellcolor[HTML]{FFCB2F}39443                    \\ \hline
\rowcolor[HTML]{FFFC9E} 
\cellcolor[HTML]{C0C0C0}75  & 44971                                                & \cellcolor[HTML]{34FF34}38111                    & \cellcolor[HTML]{FFCB2F}38650                    & 39966                                            & 41654                                            & \cellcolor[HTML]{FFCB2F}41462                    & 43910                                            & 44911                                            & 43740                                            \\ \hline
\rowcolor[HTML]{FFFC9E} 
\cellcolor[HTML]{C0C0C0}100 & \cellcolor[HTML]{FFCB2F}39659                        & 40036                                            & 40583                                            & 39383                                            & \cellcolor[HTML]{FFCB2F}41258                    & 43156                                            & \cellcolor[HTML]{FFCB2F}43066                    & \cellcolor[HTML]{FFCB2F}41932                    & 41156                                            \\ \hline
\end{tabular}
\end{table}
	
�rednia warto�� najlepszych wynik�w dla wsp�czynnika k r�wnego 50 wynosi 45 763. �rednia dla ilo�ci iteracji r�wnej 75 jest wy�sza i jest r�wna 41 931. Ostatnia pr�ba daje �redni� na poziomie 41 167. R�nica pomi�dzy 50 iteracjami a 75 wynosi 3 832, a r�nica mi�dzy 75 iteracjami a 100 jest ju� znacznie mniejsza i wynosi 764. Wyliczaj�c �redni� warto�� dla wszystkich wsp�czynnik�w parowania mo�na zauwa�y�, �e najlepsze wyniki przypadaj� dla warto�ci od 0.2 do 0.4. �rednia wynik�w jest wtedy mniejsza od 40 000.
	
W tabeli numer \ref{tab:avg_1_mrowkowy} znajduj� si� �rednie wyniki z dziesi�ciu pomiar�w dla takich samych parametr�w. Najlepsze wyniki wyst�puj� dla wsp�czynnika parowania znajduj�cego si� w zakresie od 0.1 do 0.4. Warto�ci wsp�czynnika r�wne i wy�sze od 0.5 dla ka�dej ilo�ci iteracji s� gorsze od warto�ci oryginalnej. Najlepsza �rednia pomiar�w wynosi 41 440. Wyst�pi�a ona dla k r�wnego 75 oraz wsp�czynnika parowania r�wnego 0.2.

	
\begin{table}[h!]
\label{tab:avg_1_mrowkowy}
\caption[Trasa I - �rednie wyniki dla algorytmu mr�wkowego]
{Trasa I - �rednie wyniki algorytmu mr�wkowego dla danych parametr�w wej�ciowych}
\begin{tabular}{|
>{\columncolor[HTML]{C0C0C0}}c |l|
>{\columncolor[HTML]{FFFC9E}}l |
>{\columncolor[HTML]{FFFC9E}}l |
>{\columncolor[HTML]{FFFC9E}}l |
>{\columncolor[HTML]{FFFFFF}}l |
>{\columncolor[HTML]{FFFFFF}}l |
>{\columncolor[HTML]{FFFFFF}}l |
>{\columncolor[HTML]{FFFFFF}}l |
>{\columncolor[HTML]{FFFFFF}}l |}
\hline
k   & \multicolumn{1}{c|}{\cellcolor[HTML]{C0C0C0}0.1}     & \multicolumn{1}{c|}{\cellcolor[HTML]{C0C0C0}0.2} & \multicolumn{1}{c|}{\cellcolor[HTML]{C0C0C0}0.3} & \multicolumn{1}{c|}{\cellcolor[HTML]{C0C0C0}0.4} & \multicolumn{1}{c|}{\cellcolor[HTML]{C0C0C0}0.5} & \multicolumn{1}{c|}{\cellcolor[HTML]{C0C0C0}0.6} & \multicolumn{1}{c|}{\cellcolor[HTML]{C0C0C0}0.7} & \multicolumn{1}{c|}{\cellcolor[HTML]{C0C0C0}0.8} & \multicolumn{1}{c|}{\cellcolor[HTML]{C0C0C0}0.9} \\ \hline
50  & \cellcolor[HTML]{FFFFFF}{\color[HTML]{333333} 63984} & 43885                                            & \cellcolor[HTML]{FFCB2F}43058                    & 42144                                            & 56972                                            & 58786                                            & 60393                                            & 55217                                            & \cellcolor[HTML]{FFCB2F}53567                    \\ \hline
75  & \cellcolor[HTML]{FFFC9E}48456                        & \cellcolor[HTML]{34FF34}41440                    & 46490                                            & 47186                                            & \cellcolor[HTML]{FFCB2F}52814                    & 59712                                            & \cellcolor[HTML]{FFCB2F}55389                    & 57640                                            & 56455                                            \\ \hline
100 & \cellcolor[HTML]{FFCB2F}43634                        & 42900                                            & 47014                                            & \cellcolor[HTML]{FFCB2F}41756                    & 54957                                            & \cellcolor[HTML]{FFCB2F}55704                    & 59148                                            & \cellcolor[HTML]{FFCB2F}53122                    & 54116                                            \\ \hline
\end{tabular}
\end{table}

Istotnym parametrem jaki nale�y wzi�� pod uwag� jest to, jak cz�sto algorytm jest w stanie poprawi� oryginaln� tras�. W tym celu zosta�a stworzona tabela numer \ref{tab:sum_1_mrowkowy}. Mo�emy zauwa�y� z jak� skuteczno�ci� program jest w stanie poprawi� tras�. Wyniki potwierdzaj� wcze�niejsze wnioski. Najlepsze wyniki mo�na zestawi� ze wsp�czynnikiem parowania feromon�w z zakresu od 0.1 do 0.4. Co wi�cej w zale�no�ci od ilo�ci iteracji dla ka�dego ze wsp�czynnik�w trasa zosta�a polepszona.

\begin{table}[h!]
\label{tab:sum_1_mrowkowy}
\caption[Trasa I - suma lepszych tras w por�wnaniu do oryginalnej]
{Trasa I - suma lepszych tras w por�wnaniu do oryginalnej}
\begin{tabular}{|c|l|l|l|l|l|l|l|l|l|}
\hline
\rowcolor[HTML]{C0C0C0} 
k                           & \multicolumn{1}{c|}{\cellcolor[HTML]{C0C0C0}0.1} & \multicolumn{1}{c|}{\cellcolor[HTML]{C0C0C0}0.2} & \multicolumn{1}{c|}{\cellcolor[HTML]{C0C0C0}0.3} & \multicolumn{1}{c|}{\cellcolor[HTML]{C0C0C0}0.4} & \multicolumn{1}{c|}{\cellcolor[HTML]{C0C0C0}0.5} & \multicolumn{1}{c|}{\cellcolor[HTML]{C0C0C0}0.6} & \multicolumn{1}{c|}{\cellcolor[HTML]{C0C0C0}0.7} & \multicolumn{1}{c|}{\cellcolor[HTML]{C0C0C0}0.8} & \multicolumn{1}{c|}{\cellcolor[HTML]{C0C0C0}0.9} \\ \hline
\rowcolor[HTML]{FFFFFF} 
\cellcolor[HTML]{C0C0C0}50  & {\color[HTML]{333333} 0\%}                       & \cellcolor[HTML]{C6EFCE}100\%                    & \cellcolor[HTML]{C6EFCE}100\%                    & \cellcolor[HTML]{C6EFCE}100\%                    & 20\%                                             & 10\%                                             & 0\%                                              & 40\%                                             & 30\%                                             \\ \hline
\rowcolor[HTML]{FFFFFF} 
\cellcolor[HTML]{C0C0C0}75  & 60\%                                             & \cellcolor[HTML]{C6EFCE}100\%                    & 80\%                                             & 80\%                                             & 50\%                                             & 40\%                                             & 20\%                                             & 10\%                                             & 30\%                                             \\ \hline
\rowcolor[HTML]{FFFFFF} 
\cellcolor[HTML]{C0C0C0}100 & \cellcolor[HTML]{C6EFCE}100\%                    & \cellcolor[HTML]{C6EFCE}100\%                    & 70\%                                             & \cellcolor[HTML]{C6EFCE}100\%                    & 30\%                                             & 30\%                                             & 10\%                                             & 50\%                                             & 40\%                                             \\ \hline
\end{tabular}
\end{table}
	
Na wykresie \ref{fig:kl_1} przedstawione zosta�o procentowe por�wnanie �redniej warto�ci dla ka�dego wsp�czynnika parowania i warto�ci \textit{k} r�wnej 50. Z wykresu mo�na odczyta�, �e algorytm mr�wkowy w najgorszym przypadku zadzia�a� ze �redni� o 30\% wi�ksz� od warto�ci oryginalnej. Optymalizacje natomiast si�gaj� warto�ci od 10.25\% do 13.81\%. 
	
\begin{figure}[h!]
	\centering
	\includegraphics[scale=0.75]{grafika/kl_1}
	\caption{Wykres 1}
	\label{fig:kl_1}
\end{figure}

Na wykresie \ref{fig:kl_2} zwi�kszona zosta�a ilo�� iteracji do 75 wyniki uleg�y poprawie. Najgorsza warto�� zosta�a zredukowana o ok 9\%, a najlepszy wynik doszed� do 15.25\% co jest te� najwy�sz� �redni� po�r�d wszystkich mo�liwo�ci. 
	
\begin{figure}[h!]
	\centering
	\includegraphics[scale=0.75]{grafika/kl_2}
	\caption{Wykres 1}
	\label{fig:kl_1}
\end{figure}
	
Dla ostatniej mo�liwej warto�ci iteracji warto�� najgorsza ponownie uleg�a poprawie i wynosi�a 20.96\%. Niestety najwy�sza warto�� nie zwi�kszy�a si� wzgl�dem mniejszej liczbie iteracji. Sytuacja ta zosta�a przedstawiona na wykresie \ref{fig:kl_3}
	
\begin{figure}[h!]
	\centering
	\includegraphics[scale=0.75]{grafika/kl_3}
	\caption{Wykres 1}
	\label{fig:kl_1}
\end{figure}

Opr�cz samych prac nad optymalizacjami nale�y zwr�ci� uwag� z jak� szybko�ci� algorytm jest w stanie przedstawi� rozwi�zanie. W tabeli \ref{tab:time_1_mrowkowy} przedstawiony jest �redni czas wykonania algorytmy. Oczywistym wydaje si�, �e wraz ze wzrostem ilo�ci wykonanych iteracji wzrasta czas wykonania programu. �rednio dla programu z ilo�ci� iteracji r�wn� 50 czas oczekiwania wynosi� blisko 13 sekund. Dla 75 iteracji by�o to 21.732 sekundy, a dla 100 31.547. Zestawiaj�c wyniki czasowe z procentowymi warto�ciami tras jakie zosta�y wyprodukowane przez program, mo�na zauwa�y�, �e ilo�� wi�ksza ilo�� iteracji nie musi oznacza� optymalizacji trasy.

\begin{table}[h!]
\label{tab:time_1_mrowkowy}
\caption[Trasa I - �rednie wyniki czasu wykonania dla algorytmu mr�wkowego dla parametru ilo�ci iteracji]
{Trasa I - �rednie wyniki czasu wykonania algorytmu mr�wkowego dla parametru ilo�ci iteracji}
\begin{tabular}{|
>{\columncolor[HTML]{C0C0C0}}l |l|l|l|}
\hline
iteracje & 50     & 75     & 100    \\ \hline
czas     & 12.945 & 21.732 & 31.547 \\ \hline
\end{tabular}
\end{table}

\subsection{Wyniki algorytm�w zach�annych}

\subsection{Por�wnanie algorytm�w}

\section{Trasa II}
\subsection{Wyniki algorytmu genetycznego}
\begin{table}[htb]
\caption[Trasa II -�rednie wyniki dla algorytmu genetycznego]
{Trasa II - �rednie wyniki algorytmu genetycznego dla danych parametr�w wej�ciowych}
\scalebox{0.8}{
\begin{tabular}{|l|l|l|l|l|l|l|l|l|l|l|}
\hline
\rowcolor[HTML]{C0C0C0} 
\multicolumn{1}{|c|}{\cellcolor[HTML]{C0C0C0}}                    & \multicolumn{1}{c|}{\cellcolor[HTML]{C0C0C0}}                    & \multicolumn{1}{c|}{\cellcolor[HTML]{C0C0C0}PMX}    & \multicolumn{1}{c|}{\cellcolor[HTML]{C0C0C0}PMX}     & \multicolumn{1}{c|}{\cellcolor[HTML]{C0C0C0}PMX}    & \multicolumn{1}{c|}{\cellcolor[HTML]{C0C0C0}OX}     & \multicolumn{1}{c|}{\cellcolor[HTML]{C0C0C0}OX}      & \multicolumn{1}{c|}{\cellcolor[HTML]{C0C0C0}OX}     & \multicolumn{1}{c|}{\cellcolor[HTML]{C0C0C0}CX}     & \multicolumn{1}{c|}{\cellcolor[HTML]{C0C0C0}CX}      & \multicolumn{1}{c|}{\cellcolor[HTML]{C0C0C0}CX}     \\ \cline{3-11} 
\rowcolor[HTML]{C0C0C0} 
\multicolumn{1}{|c|}{\multirow{-2}{*}{\cellcolor[HTML]{C0C0C0}N}} & \multicolumn{1}{c|}{\multirow{-2}{*}{\cellcolor[HTML]{C0C0C0}k}} & \multicolumn{1}{c|}{\cellcolor[HTML]{C0C0C0}Insert} & \multicolumn{1}{c|}{\cellcolor[HTML]{C0C0C0}Replace} & \multicolumn{1}{c|}{\cellcolor[HTML]{C0C0C0}Revert} & \multicolumn{1}{c|}{\cellcolor[HTML]{C0C0C0}Insert} & \multicolumn{1}{c|}{\cellcolor[HTML]{C0C0C0}Replace} & \multicolumn{1}{c|}{\cellcolor[HTML]{C0C0C0}Revert} & \multicolumn{1}{c|}{\cellcolor[HTML]{C0C0C0}Insert} & \multicolumn{1}{c|}{\cellcolor[HTML]{C0C0C0}Replace} & \multicolumn{1}{c|}{\cellcolor[HTML]{C0C0C0}Revert} \\ \hline
\rowcolor[HTML]{FFFFFF} 
\cellcolor[HTML]{C0C0C0}100                                       & \cellcolor[HTML]{C0C0C0}100                                      & {\color[HTML]{333333} 99575}                        & 107771                                               & 111797                                              & 113364                                              & 119707                                               & 120570                                              & 110318                                              & 116843                                               & 116749                                              \\ \hline
\rowcolor[HTML]{FFFFFF} 
\cellcolor[HTML]{C0C0C0}100                                       & \cellcolor[HTML]{C0C0C0}1000                                     & 56105                                               & 70104                                                & 88449                                               & 60788                                               & 71531                                                & 93290                                               & 59452                                               & 70589                                                & 88734                                               \\ \hline
\rowcolor[HTML]{FFFFFF} 
\cellcolor[HTML]{C0C0C0}100                                       & \cellcolor[HTML]{C0C0C0}2000                                     & 48407                                               & 61102                                                & 87698                                               & 50719                                               & 61516                                                & 92912                                               & 48634                                               & 61984                                                & 88546                                               \\ \hline
\rowcolor[HTML]{FFFFFF} 
\cellcolor[HTML]{C0C0C0}100                                       & \cellcolor[HTML]{C0C0C0}4000                                     & 47154                                               & 54452                                                & 47154                                               & 48203                                               & 55866                                                & 92145                                               & 43488                                               & 52586                                                & 86276                                               \\ \hline
\rowcolor[HTML]{FFFFFF} 
\cellcolor[HTML]{C0C0C0}1000                                      & \cellcolor[HTML]{C0C0C0}100                                      & 65248                                               & 67898                                                & 64703                                               & 78946                                               & 83318                                                & 81904                                               & {\color[HTML]{333333} 47854}                        & 46921                                                & 49695                                               \\ \hline
\rowcolor[HTML]{FFFFFF} 
\cellcolor[HTML]{C0C0C0}1000                                      & \cellcolor[HTML]{C0C0C0}1000                                     & \cellcolor[HTML]{FFFC9E}39419                       & 43935                                                & 55085                                               & 41294                                               & 46914                                                & 68463                                               & \cellcolor[HTML]{FFCB2F}30385                       & \cellcolor[HTML]{FFFC9E}32616                        & 43041                                               \\ \hline
\cellcolor[HTML]{C0C0C0}1000                                      & \cellcolor[HTML]{C0C0C0}2000                                     & \cellcolor[HTML]{FFFC9E}34882                       & \cellcolor[HTML]{FFFFFF}40914                        & \cellcolor[HTML]{FFFFFF}57467                       & \cellcolor[HTML]{FFFC9E}39237                       & \cellcolor[HTML]{FFFFFF}44605                        & \cellcolor[HTML]{FFFFFF}65532                       & \cellcolor[HTML]{FFCB2F}30631                       & \cellcolor[HTML]{FFFC9E}31273                        & \cellcolor[HTML]{FFFFFF}48346                       \\ \hline
\cellcolor[HTML]{C0C0C0}1000                                      & \cellcolor[HTML]{C0C0C0}4000                                     & \cellcolor[HTML]{FFFC9E}36160                       & \cellcolor[HTML]{FFFC9E}39848                        & \cellcolor[HTML]{FFFFFF}55774                       & \cellcolor[HTML]{FFFC9E}36414                       & \cellcolor[HTML]{FFFFFF}44856                        & \cellcolor[HTML]{FFFFFF}68215                       & \cellcolor[HTML]{FFCB2F}30443                       & \cellcolor[HTML]{FFFC9E}30720                        & \cellcolor[HTML]{FFFFFF}48295                       \\ \hline
\rowcolor[HTML]{FFFFFF} 
\cellcolor[HTML]{C0C0C0}2000                                      & \cellcolor[HTML]{C0C0C0}100                                      & 58533                                               & 59876                                                & 60898                                               & 70291                                               & 71693                                                & 72095                                               & \cellcolor[HTML]{FFFC9E}38655                       & \cellcolor[HTML]{FFFC9E}39129                        & \cellcolor[HTML]{FFCB2F}37204                       \\ \hline
\cellcolor[HTML]{C0C0C0}2000                                      & \cellcolor[HTML]{C0C0C0}1000                                     & \cellcolor[HTML]{FFFC9E}34939                       & \cellcolor[HTML]{FFFC9E}38272                        & \cellcolor[HTML]{FFFFFF}49381                       & \cellcolor[HTML]{FFFC9E}35951                       & \cellcolor[HTML]{FFFC9E}39779                        & \cellcolor[HTML]{FFFFFF}62656                       & \cellcolor[HTML]{34FF34}28196                       & \cellcolor[HTML]{FFFFFF}29415                        & \cellcolor[HTML]{FFFFFF}33519                       \\ \hline
\cellcolor[HTML]{C0C0C0}2000                                      & \cellcolor[HTML]{C0C0C0}2000                                     & \cellcolor[HTML]{FFFC9E}32900                       & \cellcolor[HTML]{FFFC9E}39068                        & \cellcolor[HTML]{FFFFFF}48607                       & \cellcolor[HTML]{FFFC9E}34474                       & \cellcolor[HTML]{FFFFFF}41349                        & \cellcolor[HTML]{FFFFFF}56907                       & \cellcolor[HTML]{FFFC9E}28802                       & \cellcolor[HTML]{FFCB2F}28506                        & \cellcolor[HTML]{FFFC9E}34458                       \\ \hline
\rowcolor[HTML]{FFFC9E} 
\cellcolor[HTML]{C0C0C0}2000                                      & \cellcolor[HTML]{C0C0C0}4000                                     & 32626                                               & 36244                                                & \cellcolor[HTML]{FFFFFF}49132                       & 36333                                               & 39164                                                & \cellcolor[HTML]{FFFFFF}57952                       & 28808                                               & \cellcolor[HTML]{FFCB2F}28453                        & 32683                                               \\ \hline
\end{tabular}}
\end{table}


\begin{table}[]
\caption[Trasa II - najlepsze wyniki dla algorytmu genetycznego]
{Trasa II - najlepsze wyniki algorytmu genetycznego dla danych parametr�w wej�ciowych}
\scalebox{0.8}{
\begin{tabular}{|l|l|l|l|l|l|l|l|l|l|l|}
\hline
\rowcolor[HTML]{C0C0C0} 
\multicolumn{1}{|c|}{\cellcolor[HTML]{C0C0C0}}                    & \multicolumn{1}{c|}{\cellcolor[HTML]{C0C0C0}}                    & PMX                           & PMX                           & PMX                           & OX                            & OX                            & OX                            & CX                                                   & CX                            & CX                            \\ \cline{3-11} 
\rowcolor[HTML]{C0C0C0} 
\multicolumn{1}{|c|}{\multirow{-2}{*}{\cellcolor[HTML]{C0C0C0}N}} & \multicolumn{1}{c|}{\multirow{-2}{*}{\cellcolor[HTML]{C0C0C0}k}} & Insert                        & Replace                       & Revert                        & Insert                        & Replace                       & Revert                        & Insert                                               & Replace                       & Revert                        \\ \hline
\rowcolor[HTML]{FFFFFF} 
\cellcolor[HTML]{C0C0C0}100                                       & \cellcolor[HTML]{C0C0C0}100                                      & {\color[HTML]{333333} 89707}  & 95824                         & 100856                        & 102788                        & 110057                        & 114491                        & 104749                                               & 104765                        & 101883                        \\ \hline
\rowcolor[HTML]{FFFFFF} 
\cellcolor[HTML]{C0C0C0}100                                       & \cellcolor[HTML]{C0C0C0}1000                                     & 44863                         & 63868                         & 76641                         & 54433                         & 59588                         & 80783                         & 53688                                                & 55723                         & 75815                         \\ \hline
\rowcolor[HTML]{FFFFFF} 
\cellcolor[HTML]{C0C0C0}100                                       & \cellcolor[HTML]{C0C0C0}2000                                     & 42084                         & 56000                         & 79817                         & 46908                         & 54427                         & 86609                         & 43440                                                & 48834                         & 80426                         \\ \hline
\rowcolor[HTML]{FFFFFF} 
\cellcolor[HTML]{C0C0C0}100                                       & \cellcolor[HTML]{C0C0C0}4000                                     & \cellcolor[HTML]{FFFC9E}40309 & 49246                         & 40309                         & 43816                         & 49864                         & 78281                         & \cellcolor[HTML]{FFCB2F}37172                        & 47158                         & 78780                         \\ \hline
\rowcolor[HTML]{FFFFFF} 
\cellcolor[HTML]{C0C0C0}1000                                      & \cellcolor[HTML]{C0C0C0}100                                      & 53565                         & 59523                         & 59419                         & 71181                         & 75567                         & 73827                         & \cellcolor[HTML]{FFFC9E}{\color[HTML]{333333} 40068} & \cellcolor[HTML]{FFFC9E}38220 & \cellcolor[HTML]{FFCB2F}33442 \\ \hline
\cellcolor[HTML]{C0C0C0}1000                                      & \cellcolor[HTML]{C0C0C0}1000                                     & \cellcolor[HTML]{FFFC9E}34352 & \cellcolor[HTML]{FFFC9E}38069 & \cellcolor[HTML]{FFFFFF}48381 & \cellcolor[HTML]{FFFC9E}34282 & \cellcolor[HTML]{FFFFFF}44087 & \cellcolor[HTML]{FFFFFF}62194 & \cellcolor[HTML]{FFCB2F}28947                        & \cellcolor[HTML]{FFFC9E}29655 & \cellcolor[HTML]{FFFC9E}34089 \\ \hline
\cellcolor[HTML]{C0C0C0}1000                                      & \cellcolor[HTML]{C0C0C0}2000                                     & \cellcolor[HTML]{FFFC9E}32942 & \cellcolor[HTML]{FFFC9E}36491 & \cellcolor[HTML]{FFFFFF}48353 & \cellcolor[HTML]{FFFC9E}33013 & \cellcolor[HTML]{FFFFFF}40481 & \cellcolor[HTML]{FFFFFF}58856 & \cellcolor[HTML]{FFFC9E}28047                        & \cellcolor[HTML]{FFCB2F}27858 & \cellcolor[HTML]{FFFC9E}35650 \\ \hline
\rowcolor[HTML]{FFFC9E} 
\cellcolor[HTML]{C0C0C0}1000                                      & \cellcolor[HTML]{C0C0C0}4000                                     & 33021                         & 33720                         & \cellcolor[HTML]{FFFFFF}47438 & 29602                         & 38640                         & \cellcolor[HTML]{FFFFFF}57355 & 28442                                                & \cellcolor[HTML]{FFCB2F}28027 & 35271                         \\ \hline
\rowcolor[HTML]{FFFFFF} 
\cellcolor[HTML]{C0C0C0}2000                                      & \cellcolor[HTML]{C0C0C0}100                                      & 50007                         & 53464                         & 51193                         & 61228                         & 67171                         & 59619                         & \cellcolor[HTML]{FFFC9E}33775                        & \cellcolor[HTML]{FFFC9E}35063 & \cellcolor[HTML]{FFCB2F}31852 \\ \hline
\rowcolor[HTML]{FFFC9E} 
\cellcolor[HTML]{C0C0C0}2000                                      & \cellcolor[HTML]{C0C0C0}1000                                     & 29758                         & 33420                         & \cellcolor[HTML]{FFFFFF}43562 & 32178                         & 33255                         & \cellcolor[HTML]{FFFFFF}56070 & \cellcolor[HTML]{34FF34}27655                        & 27910                         & 29839                         \\ \hline
\rowcolor[HTML]{FFFC9E} 
\cellcolor[HTML]{C0C0C0}2000                                      & \cellcolor[HTML]{C0C0C0}2000                                     & 31390                         & 35329                         & 38151                         & 30047                         & 31229                         & \cellcolor[HTML]{FFFFFF}48944 & 28178                                                & \cellcolor[HTML]{FFCB2F}27381 & 29839                         \\ \hline
\rowcolor[HTML]{FFFC9E} 
\cellcolor[HTML]{C0C0C0}2000                                      & \cellcolor[HTML]{C0C0C0}4000                                     & 30080                         & 31724                         & 38944                         & 32360                         & 35025                         & \cellcolor[HTML]{FFFFFF}49517 & 28177                                                & \cellcolor[HTML]{FFCB2F}27925 & 29840                         \\ \hline
\end{tabular}}
\end{table}

\begin{figure}[h]
	\centering
	\includegraphics[scale=0.4]{grafika/Trasa2GA.png}
	\caption{Trasa II - Zale�no�� �redniej odleg�o�ci od czasu}
	\label{mutacja}
\end{figure}	

\subsection{Wyniki algorytmu mr�wkowego}

AVG

% Please add the following required packages to your document preamble:
% \usepackage[table,xcdraw]{xcolor}
% If you use beamer only pass "xcolor=table" option, i.e. \documentclass[xcolor=table]{beamer}
\begin{table}[]
\caption[Trasa II - �rednie wyniki dla algorytmu mr�wkowego]
{Trasa II - �rednie wyniki algorytmu mr�wkowego dla danych parametr�w wej�ciowych}
\begin{tabular}{|c|l|l|l|l|l|l|l|l|l|}
\hline
\rowcolor[HTML]{C0C0C0} 
k                           & \multicolumn{1}{c|}{\cellcolor[HTML]{C0C0C0}0.1} & \multicolumn{1}{c|}{\cellcolor[HTML]{C0C0C0}0.2} & \multicolumn{1}{c|}{\cellcolor[HTML]{C0C0C0}0.3} & \multicolumn{1}{c|}{\cellcolor[HTML]{C0C0C0}0.4} & \multicolumn{1}{c|}{\cellcolor[HTML]{C0C0C0}0.5} & \multicolumn{1}{c|}{\cellcolor[HTML]{C0C0C0}0.6} & \multicolumn{1}{c|}{\cellcolor[HTML]{C0C0C0}0.7} & \multicolumn{1}{c|}{\cellcolor[HTML]{C0C0C0}0.8} & \multicolumn{1}{c|}{\cellcolor[HTML]{C0C0C0}0.9} \\ \hline
\rowcolor[HTML]{FFFFFF} 
\cellcolor[HTML]{C0C0C0}50  & {\color[HTML]{333333} 105951}                    & 90688                                            & \cellcolor[HTML]{34FF34}77772                    & 82462                                            & 80346                                            & \cellcolor[HTML]{FFCB2F}81396                    & 81451                                            & 80728                                            & \cellcolor[HTML]{FFCB2F}80210                    \\ \hline
\rowcolor[HTML]{FFFFFF} 
\cellcolor[HTML]{C0C0C0}75  & 94734                                            & 83801                                            & 78304                                            & 82241                                            & \cellcolor[HTML]{FFCB2F}78774                    & 81767                                            & 80815                                            & 80590                                            & 80721                                            \\ \hline
\rowcolor[HTML]{FFCB2F} 
\cellcolor[HTML]{C0C0C0}100 & 90821                                            & 83212                                            & \cellcolor[HTML]{FFFFFF}78372                    & 80373                                            & \cellcolor[HTML]{FFFFFF}79902                    & \cellcolor[HTML]{FFFFFF}81995                    & 79855                                            & 80240                                            & \cellcolor[HTML]{FFFFFF}80458                    \\ \hline
\end{tabular}
\end{table}


BEST

% Please add the following required packages to your document preamble:
% \usepackage[table,xcdraw]{xcolor}
% If you use beamer only pass "xcolor=table" option, i.e. \documentclass[xcolor=table]{beamer}
\begin{table}[]
\caption[Trasa II - najlepsze wyniki dla algorytmu mr�wkowego]
{Trasa II - najlepsze wyniki algorytmu mr�wkowego dla danych parametr�w wej�ciowych}
\begin{tabular}{|c|l|l|l|l|l|l|l|l|l|}
\hline
\rowcolor[HTML]{C0C0C0} 
k                           & \multicolumn{1}{c|}{\cellcolor[HTML]{C0C0C0}0.1}      & \multicolumn{1}{c|}{\cellcolor[HTML]{C0C0C0}0.2} & \multicolumn{1}{c|}{\cellcolor[HTML]{C0C0C0}0.3} & \multicolumn{1}{c|}{\cellcolor[HTML]{C0C0C0}0.4} & \multicolumn{1}{c|}{\cellcolor[HTML]{C0C0C0}0.5} & \multicolumn{1}{c|}{\cellcolor[HTML]{C0C0C0}0.6} & \multicolumn{1}{c|}{\cellcolor[HTML]{C0C0C0}0.7} & \multicolumn{1}{c|}{\cellcolor[HTML]{C0C0C0}0.8} & \multicolumn{1}{c|}{\cellcolor[HTML]{C0C0C0}0.9} \\ \hline
\cellcolor[HTML]{C0C0C0}50  & \cellcolor[HTML]{FFFFFF}{\color[HTML]{333333} 101010} & \cellcolor[HTML]{FFFFFF}86933                    & \cellcolor[HTML]{FFCB2F}75226                    & \cellcolor[HTML]{FFFFFF}78697                    & \cellcolor[HTML]{34FF34}75069                    & \cellcolor[HTML]{FFCB2F}77189                    & \cellcolor[HTML]{FFFFFF}77539                    & \cellcolor[HTML]{FFCB2F}76637                    & \cellcolor[HTML]{FFCB2F}76199                    \\ \hline
\rowcolor[HTML]{FFFFFF} 
\cellcolor[HTML]{C0C0C0}75  & 91453                                                 & \cellcolor[HTML]{FFCB2F}81128                    & 75485                                            & 78774                                            & 76617                                            & 78815                                            & 76810                                            & 77423                                            & 76446                                            \\ \hline
\rowcolor[HTML]{FFFFFF} 
\cellcolor[HTML]{C0C0C0}100 & \cellcolor[HTML]{FFCB2F}85587                         & 81400                                            & 76782                                            & \cellcolor[HTML]{FFCB2F}76559                    & 75351                                            & 77897                                            & \cellcolor[HTML]{FFCB2F}75233                    & 76713                                            & 77081                                            \\ \hline
\end{tabular}
\end{table}

czas

% Please add the following required packages to your document preamble:
% \usepackage[table,xcdraw]{xcolor}
% If you use beamer only pass "xcolor=table" option, i.e. \documentclass[xcolor=table]{beamer}
\begin{table}[]
\caption[Trasa II - �rednie wyniki czasu wykonania dla algorytmu mr�wkowego dla parametru ilo�ci iteracji]
{Trasa II - �rednie wyniki czasu wykonania algorytmu mr�wkowego dla parametru ilo�ci iteracji}
\begin{tabular}{|
>{\columncolor[HTML]{C0C0C0}}l |l|l|l|}
\hline
iteracje & 50     & 75     & 100    \\ \hline
czas     & 28.325 & 42.689 & 60.281 \\ \hline
\end{tabular}
\end{table}



\subsection{Wyniki algorytm�w zach�annych}

\subsection{Por�wnanie algorytm�w}

\section{Trasa III}

\subsection{Wyniki algorytmu genetycznego}

% Please add the following required packages to your document preamble:
% \usepackage{multirow}
% \usepackage[table,xcdraw]{xcolor}
% If you use beamer only pass "xcolor=table" option, i.e. \documentclass[xcolor=table]{beamer}
\begin{table}[]
\caption[Trasa III -�rednie wyniki dla algorytmu genetycznego]
{Trasa III - �rednie wyniki algorytmu genetycznego dla danych parametr�w wej�ciowych}
\scalebox{0.8}{
\begin{tabular}{|l|l|l|l|l|l|l|l|l|l|l|}
\hline
\rowcolor[HTML]{C0C0C0} 
\multicolumn{1}{|c|}{\cellcolor[HTML]{C0C0C0}}                    & \multicolumn{1}{c|}{\cellcolor[HTML]{C0C0C0}}                    & PMX                           & PMX    & PMX    & OX     & OX     & OX     & CX                            & CX                            & CX     \\ \cline{3-11} 
\rowcolor[HTML]{C0C0C0} 
\multicolumn{1}{|c|}{\multirow{-2}{*}{\cellcolor[HTML]{C0C0C0}N}} & \multicolumn{1}{c|}{\multirow{-2}{*}{\cellcolor[HTML]{C0C0C0}k}} & Insert                        & Swap   & Revert & Insert & Swap   & Revert & Insert                        & Swap                          & Revert \\ \hline
\rowcolor[HTML]{FFFFFF} 
\cellcolor[HTML]{C0C0C0}100                                       & \cellcolor[HTML]{C0C0C0}100                                      & 202200                        & 215800 & 223134 & 226673 & 238350 & 244471 & 226996                        & 238909                        & 248130 \\ \hline
\rowcolor[HTML]{FFFFFF} 
\cellcolor[HTML]{C0C0C0}100                                       & \cellcolor[HTML]{C0C0C0}1000                                     & 101445                        & 119726 & 156991 & 108762 & 130635 & 168963 & 121076                        & 140420                        & 170028 \\ \hline
\rowcolor[HTML]{FFFFFF} 
\cellcolor[HTML]{C0C0C0}100                                       & \cellcolor[HTML]{C0C0C0}2000                                     & 88501                         & 112378 & 149298 & 95249  & 112897 & 159348 & 95405                         & 110801                        & 162912 \\ \hline
\rowcolor[HTML]{FFFFFF} 
\cellcolor[HTML]{C0C0C0}100                                       & \cellcolor[HTML]{C0C0C0}4000                                     & 81056                         & 97882  & 151635 & 80120  & 99671  & 163374 & 74624                         & 97972                         & 153400 \\ \hline
\rowcolor[HTML]{FFFFFF} 
\cellcolor[HTML]{C0C0C0}1000                                      & \cellcolor[HTML]{C0C0C0}100                                      & 138988                        & 141411 & 138102 & 159437 & 165202 & 161127 & {\color[HTML]{333333} 100709} & 114035                        & 108668 \\ \hline
\rowcolor[HTML]{FFFFFF} 
\cellcolor[HTML]{C0C0C0}1000                                      & \cellcolor[HTML]{C0C0C0}1000                                     & 60548                         & 66745  & 93017  & 70666  & 76984  & 117566 & \cellcolor[HTML]{FFCB2F}43728 & 54573                         & 85516  \\ \hline
\rowcolor[HTML]{FFFFFF} 
\cellcolor[HTML]{C0C0C0}1000                                      & \cellcolor[HTML]{C0C0C0}2000                                     & 53736                         & 68402  & 96311  & 59332  & 71183  & 117932 & \cellcolor[HTML]{FFCB2F}42545 & \cellcolor[HTML]{FFFC9E}50305 & 90144  \\ \hline
\rowcolor[HTML]{FFFFFF} 
\cellcolor[HTML]{C0C0C0}1000                                      & \cellcolor[HTML]{C0C0C0}4000                                     & \cellcolor[HTML]{FFFC9E}49323 & 66474  & 96935  & 56877  & 69448  & 118314 & \cellcolor[HTML]{FFCB2F}40121 & \cellcolor[HTML]{FFFC9E}48213 & 86232  \\ \hline
\rowcolor[HTML]{FFFFFF} 
\cellcolor[HTML]{C0C0C0}2000                                      & \cellcolor[HTML]{C0C0C0}100                                      & 126626                        & 127066 & 125408 & 147299 & 148430 & 147276 & 64776                         & 89803                         & 81523  \\ \hline
\rowcolor[HTML]{FFFFFF} 
\cellcolor[HTML]{C0C0C0}2000                                      & \cellcolor[HTML]{C0C0C0}1000                                     & 54123                         & 64199  & 82598  & 57743  & 68314  & 103521 & \cellcolor[HTML]{FFCB2F}34365 & \cellcolor[HTML]{FFFC9E}39530 & 50624  \\ \hline
\rowcolor[HTML]{FFFFFF} 
\cellcolor[HTML]{C0C0C0}2000                                      & \cellcolor[HTML]{C0C0C0}2000                                     & \cellcolor[HTML]{FFFC9E}45893 & 55922  & 82523  & 51700  & 63697  & 96266  & \cellcolor[HTML]{FFCB2F}36518 & \cellcolor[HTML]{FFFC9E}39200 & 59014  \\ \hline
\rowcolor[HTML]{FFFFFF} 
\cellcolor[HTML]{C0C0C0}2000                                      & \cellcolor[HTML]{C0C0C0}4000                                     & \cellcolor[HTML]{FFFC9E}47206 & 57436  & 81653  & 53149  & 66321  & 96440  & \cellcolor[HTML]{34FF34}34168 & \cellcolor[HTML]{FFFC9E}37951 & 56645  \\ \hline
\end{tabular}}
\end{table}

\begin{table}[]
\caption[Trasa III - najlepsze wyniki dla algorytmu genetycznego]
{Trasa III - najlepsze wyniki algorytmu genetycznego dla danych parametr�w wej�ciowych}
\scalebox{0.8}{
\begin{tabular}{|
>{\columncolor[HTML]{C0C0C0}}l |
>{\columncolor[HTML]{C0C0C0}}l |
>{\columncolor[HTML]{FFFFFF}}l |
>{\columncolor[HTML]{FFFFFF}}l |
>{\columncolor[HTML]{FFFFFF}}l |
>{\columncolor[HTML]{FFFFFF}}l |
>{\columncolor[HTML]{FFFFFF}}l |
>{\columncolor[HTML]{FFFFFF}}l |l|l|
>{\columncolor[HTML]{FFFFFF}}l |}
\hline
\multicolumn{1}{|c|}{\cellcolor[HTML]{C0C0C0}}                    & \multicolumn{1}{c|}{\cellcolor[HTML]{C0C0C0}}                    & \cellcolor[HTML]{C0C0C0}PMX    & \cellcolor[HTML]{C0C0C0}PMX   & \cellcolor[HTML]{C0C0C0}PMX    & \cellcolor[HTML]{C0C0C0}OX     & \cellcolor[HTML]{C0C0C0}OX    & \cellcolor[HTML]{C0C0C0}OX     & \cellcolor[HTML]{C0C0C0}CX                           & \cellcolor[HTML]{C0C0C0}CX     & \cellcolor[HTML]{C0C0C0}CX     \\ \cline{3-11} 
\multicolumn{1}{|c|}{\multirow{-2}{*}{\cellcolor[HTML]{C0C0C0}N}} & \multicolumn{1}{c|}{\multirow{-2}{*}{\cellcolor[HTML]{C0C0C0}k}} & \cellcolor[HTML]{C0C0C0}Insert & \cellcolor[HTML]{C0C0C0}Swap  & \cellcolor[HTML]{C0C0C0}Revert & \cellcolor[HTML]{C0C0C0}Insert & \cellcolor[HTML]{C0C0C0}Swap  & \cellcolor[HTML]{C0C0C0}Revert & \cellcolor[HTML]{C0C0C0}Insert                       & \cellcolor[HTML]{C0C0C0}Swap   & \cellcolor[HTML]{C0C0C0}Revert \\ \hline
\cellcolor[HTML]{C0C0C0}100                                       & \cellcolor[HTML]{C0C0C0}100                                      & 178235                         & 179169                        & 210255                         & 214392                         & 220255                        & 225128                         & \cellcolor[HTML]{FFFFFF}199164                       & \cellcolor[HTML]{FFFFFF}209900 & 228851                         \\ \hline
100                                                               & 1000                                                             & 92035                          & 97541                         & 145836                         & 92447                          & 119218                        & 156940                         & \cellcolor[HTML]{FFFFFF}108633                       & \cellcolor[HTML]{FFFFFF}134528 & 147851                         \\ \hline
100                                                               & 2000                                                             & 75472                          & 99019                         & 133028                         & 82103                          & 100138                        & 150643                         & \cellcolor[HTML]{FFFFFF}82086                        & \cellcolor[HTML]{FFFFFF}100030 & 146323                         \\ \hline
100                                                               & 4000                                                             & 66386                          & 84653                         & 128228                         & 67843                          & 87629                         & 150569                         & \cellcolor[HTML]{FFFFFF}64530                        & \cellcolor[HTML]{FFFFFF}85814  & 128407                         \\ \hline
1000                                                              & 100                                                              & 130408                         & 129829                        & 113771                         & 146677                         & 146199                        & 146851                         & \cellcolor[HTML]{FFFFFF}{\color[HTML]{333333} 80069} & \cellcolor[HTML]{FFFFFF}80657  & 91467                          \\ \hline
1000                                                              & 1000                                                             & 54442                          & 57812                         & 79987                          & 65279                          & 70631                         & 98505                          & \cellcolor[HTML]{FFCB2F}37536                        & \cellcolor[HTML]{FFFC9E}44330  & 70152                          \\ \hline
1000                                                              & 2000                                                             & \cellcolor[HTML]{FFFC9E}43708  & 57296                         & 67415                          & \cellcolor[HTML]{FFFC9E}51213  & 57795                         & 97761                          & \cellcolor[HTML]{FFCB2F}36381                        & \cellcolor[HTML]{FFFFC7}39559  & 64270                          \\ \hline
1000                                                              & 4000                                                             & \cellcolor[HTML]{FFFC9E}43938  & 57678                         & 81284                          & \cellcolor[HTML]{FFFC9E}47695  & 60624                         & 102346                         & \cellcolor[HTML]{FFCB2F}34365                        & \cellcolor[HTML]{FFFFC7}38618  & 69499                          \\ \hline
2000                                                              & 100                                                              & 117579                         & 114351                        & 112117                         & 133323                         & 133203                        & 140060                         & \cellcolor[HTML]{FFFFFF}58239                        & \cellcolor[HTML]{FFFFFF}66450  & 64827                          \\ \hline
2000                                                              & 1000                                                             & \cellcolor[HTML]{FFFC9E}46490  & \cellcolor[HTML]{FFFC9E}47507 & 72252                          & \cellcolor[HTML]{FFFC9E}50359  & 60606                         & 91474                          & \cellcolor[HTML]{34FF34}32374                        & \cellcolor[HTML]{FFFC9E}33408  & \cellcolor[HTML]{FFFC9E}41138  \\ \hline
2000                                                              & 2000                                                             & \cellcolor[HTML]{FFFC9E}42326  & \cellcolor[HTML]{FFFC9E}44568 & 70413                          & \cellcolor[HTML]{FFFC9E}45282  & \cellcolor[HTML]{FFFC9E}53114 & 81577                          & \cellcolor[HTML]{FFCB2F}33070                        & \cellcolor[HTML]{FFFC9E}36492  & \cellcolor[HTML]{FFFC9E}41724  \\ \hline
2000                                                              & 4000                                                             & \cellcolor[HTML]{FFFC9E}43332  & \cellcolor[HTML]{FFFC9E}49969 & 69794                          & \cellcolor[HTML]{FFFC9E}46970  & 56971                         & 83286                          & \cellcolor[HTML]{FFCB2F}32478                        & \cellcolor[HTML]{FFFC9E}33962  & \cellcolor[HTML]{FFFC9E}41975  \\ \hline
\end{tabular}}
\end{table}

\begin{figure}[h]
	\centering
	\includegraphics[scale=0.4]{grafika/Trasa3GA.png}
	\caption{Trasa III - Zale�no�� �redniej odleg�o�ci od czasu}
	\label{mutacja}
\end{figure}	

\subsection{Wyniki algorytmu mr�wkowego}

AVG

% Please add the following required packages to your document preamble:
% \usepackage[table,xcdraw]{xcolor}
% If you use beamer only pass "xcolor=table" option, i.e. \documentclass[xcolor=table]{beamer}
\begin{table}[]
\caption[Trasa III - �rednie wyniki dla algorytmu mr�wkowego]
{Trasa III - �rednie wyniki algorytmu mr�wkowego dla danych parametr�w wej�ciowych}
\begin{tabular}{|c|l|l|l|l|l|l|l|l|l|}
\hline
\rowcolor[HTML]{C0C0C0} 
k                           & \multicolumn{1}{c|}{\cellcolor[HTML]{C0C0C0}0.1}     & \multicolumn{1}{c|}{\cellcolor[HTML]{C0C0C0}0.2} & \multicolumn{1}{c|}{\cellcolor[HTML]{C0C0C0}0.3} & \multicolumn{1}{c|}{\cellcolor[HTML]{C0C0C0}0.4} & \multicolumn{1}{c|}{\cellcolor[HTML]{C0C0C0}0.5} & \multicolumn{1}{c|}{\cellcolor[HTML]{C0C0C0}0.6} & \multicolumn{1}{c|}{\cellcolor[HTML]{C0C0C0}0.7} & \multicolumn{1}{c|}{\cellcolor[HTML]{C0C0C0}0.8} & \multicolumn{1}{c|}{\cellcolor[HTML]{C0C0C0}0.9} \\ \hline
\rowcolor[HTML]{FFFC9E} 
\cellcolor[HTML]{C0C0C0}50  & \cellcolor[HTML]{FFFFFF}{\color[HTML]{333333} 61628} & 45302                                            & \cellcolor[HTML]{FFCB2F}43699                    & \cellcolor[HTML]{34FF34}42668                    & 48201                                            & \cellcolor[HTML]{FFCB2F}47483                    & 50840                                            & 47157                                            & 48293                                            \\ \hline
\rowcolor[HTML]{FFFC9E} 
\cellcolor[HTML]{C0C0C0}75  & 49664                                                & \cellcolor[HTML]{FFCB2F}43667                    & 44621                                            & 44292                                            & 46181                                            & 48248                                            & 48021                                            & 48112                                            & 48609                                            \\ \hline
\rowcolor[HTML]{FFCB2F} 
\cellcolor[HTML]{C0C0C0}100 & 45504                                                & \cellcolor[HTML]{FFFC9E}45802                    & \cellcolor[HTML]{FFFC9E}46286                    & \cellcolor[HTML]{FFFC9E}45465                    & 45721                                            & \cellcolor[HTML]{FFFC9E}49379                    & 47921                                            & 46410                                            & 47433                                            \\ \hline
\end{tabular}
\end{table}

BEST

% Please add the following required packages to your document preamble:
% \usepackage[table,xcdraw]{xcolor}
% If you use beamer only pass "xcolor=table" option, i.e. \documentclass[xcolor=table]{beamer}
\begin{table}[]
\caption[Trasa III - najlepsze wyniki dla algorytmu mr�wkowego]
{Trasa III - najlepsze wyniki algorytmu mr�wkowego dla danych parametr�w wej�ciowych}
\begin{tabular}{|
>{\columncolor[HTML]{C0C0C0}}c |l|
>{\columncolor[HTML]{FFFC9E}}l |
>{\columncolor[HTML]{FFFC9E}}l |
>{\columncolor[HTML]{FFFC9E}}l |
>{\columncolor[HTML]{FFFC9E}}l |
>{\columncolor[HTML]{FFFC9E}}l |
>{\columncolor[HTML]{FFFC9E}}l |
>{\columncolor[HTML]{FFFC9E}}l |
>{\columncolor[HTML]{FFFC9E}}l |}
\hline
k   & \multicolumn{1}{c|}{\cellcolor[HTML]{C0C0C0}0.1}     & \multicolumn{1}{c|}{\cellcolor[HTML]{C0C0C0}0.2} & \multicolumn{1}{c|}{\cellcolor[HTML]{C0C0C0}0.3} & \multicolumn{1}{c|}{\cellcolor[HTML]{C0C0C0}0.4} & \multicolumn{1}{c|}{\cellcolor[HTML]{C0C0C0}0.5} & \multicolumn{1}{c|}{\cellcolor[HTML]{C0C0C0}0.6} & \multicolumn{1}{c|}{\cellcolor[HTML]{C0C0C0}0.7} & \multicolumn{1}{c|}{\cellcolor[HTML]{C0C0C0}0.8} & \multicolumn{1}{c|}{\cellcolor[HTML]{C0C0C0}0.9} \\ \hline
50  & \cellcolor[HTML]{FFFFFF}{\color[HTML]{333333} 59369} & 43287                                            & \cellcolor[HTML]{FFCB2F}40214                    & \cellcolor[HTML]{FFCB2F}41158                    & 44252                                            & 43387                                            & 48045                                            & \cellcolor[HTML]{FFCB2F}41331                    & \cellcolor[HTML]{FFCB2F}41858                    \\ \hline
75  & \cellcolor[HTML]{FFFC9E}46953                        & 42126                                            & 41568                                            & 41313                                            & 43466                                            & 43728                                            & 45216                                            & 45947                                            & 46255                                            \\ \hline
100 & \cellcolor[HTML]{34FF34}40410                        & \cellcolor[HTML]{FFCB2F}41250                    & 40509                                            & 41796                                            & \cellcolor[HTML]{FFCB2F}40815                    & \cellcolor[HTML]{FFCB2F}42413                    & \cellcolor[HTML]{FFCB2F}44712                    & 42284                                            & 43308                                            \\ \hline
\end{tabular}
\end{table}

czas
% Please add the following required packages to your document preamble:
% \usepackage[table,xcdraw]{xcolor}
% If you use beamer only pass "xcolor=table" option, i.e. \documentclass[xcolor=table]{beamer}
\begin{table}[]
\caption[Trasa III - �rednie wyniki czasu wykonania dla algorytmu mr�wkowego dla parametru ilo�ci iteracji]
{Trasa III - �rednie wyniki czasu wykonania algorytmu mr�wkowego dla parametru ilo�ci iteracji}
\begin{tabular}{|
>{\columncolor[HTML]{C0C0C0}}l |l|l|l|}
\hline
iteracje & 50     & 75     & 100    \\ \hline
czas     & 84.451 & 131.146 & 176.874 \\ \hline
\end{tabular}
\end{table}



\subsection{Wyniki algorytm�w zach�annych}

\subsection{Por�wnanie algorytm�w}

\chapter*{Podsumowanie -- KL, PN, PS}
\addcontentsline{toc}{chapter}{Podsumowanie}

Cel pracy zosta� osi�gni�ty, wykonana zosta�a optymalizacja tras przejazdu ci�ar�wek z odpadami komunalnymi. Wykorzystane rozwi�zania s� dedykowane do prac nad optymalizacj� problemu komiwoja�era. Dla ka�dego z algorytm�w nale�a�o znale�� takie parametry wej�ciowe, aby optymalizacje d�ugo�ci tras by�y jak najlepsze.

Dzi�ki wykorzystaniu realnych danych do optymalizacji �cie�ek przejazd�w ci�ar�wek z odpadami komunalnymi mo�na zbada� oraz zaproponowa� rozwi�zanie problemu. Dane te zosta�y udost�pnione przez miasto Bia�ystok i wymaga�y odpowiedniego przetworzenia, w celu przeprowadzenia bada�. Wybrane �cie�ki stanowi� faktyczne trasy jakie by�y przebywane. Wybrane trzy losowo trasy sk�ada�y si� odpowiednio z 111, 150 oraz 248 punkt�w. Dzi�ki udost�pnionym rozwi�zaniu przez firm� Google, zosta�y wyliczone rzeczywiste odleg�o�ci pomi�dzy wszystkimi punktami.

Do optymalizacji wybrane zosta�y algorytmy reprezentuj�ce trzy r�ne podej�cia. Algorytm genetyczny, kt�ry zosta� opisany i zbadany przez Paw�a Stypu�kowskiego. Kolejnym algorytmem wybranym by� algorytm mr�wkowy autorstwa Kamila ��towskiego. Jako ostatnie zosta�o zaproponowanych kilka algorytm�w zach�annych przez Przemys�awa Noskowicza.

Algorytm genetyczny oraz mr�wkowy dla ka�dej mo�liwej konfiguracji parametr�w wej�ciowych zosta� uruchomiony dziesi�� razy. Algorytm genetyczny zosta� zbadany dla trzech populacji $N$: 100, 1000 oraz 2000; czterech r�nych warto�ci iteracji $k$: 100, 1000, 2000 oraz 4000; trzech r�nych krzy�owa�: $PMX$, $OX$ oraz $CX$; trzech r�nych mutacji: $Reverse$, $Swap$ oraz $Insert$. Wykonanie wszystkich bada� dla jednej trasy trwa�o kilka godzin. W algorytmie mr�wkowym parametrami wej�ciowymi by�a liczba iteracji oraz wsp�czynnik parowania feromon�w. Wykonanych by�o 50, 75 lub 100 iteracji a wsp�czynnik zakresu by� rozpoczyna� si� od 0.1 i by� zwi�kszany do 0.9 co 0.1. W ten spos�b dla jednej trasy by�o wykonanych 270 bada�. Przedstawicielami algorytm�w zach�annych u�ytych w badaniach by�y algorytm najbli�szego s�siada, algorytm najmniejszej kraw�dzi oraz algorytm A*. W celu rozwa�enia wi�kszej liczby tras zastosowano podczas dokonywania wybor�w wsp�czynnik b��du. Wsp�czynnik b��du by� uruchamiany na trzech r�nych etapach budowania docelowej trasy. Utworzone trasy zosta�y zoptymalizowane metod� 2-opt.

Po przeprowadzeniu bada� z pewno�ci� mo�na powiedzie�, �e pierwotne trasy nie by�y optymalne. Om�wione rozwi�zania by�y w stanie poprawi� oryginaln� tras� nawet o 40\%. Ka�dy z algorytm�w by� w stanie wygenerowa� wiele kr�tszych tras.

Dla pierwszej trasy, o oryginalnej d�ugo�ci 48897 metr�w, najkr�tsz� drog� o d�ugo�ci 2713 metr�w w czasie 21.73 sekund wyznaczy� algorytm mr�wkowy. Wynik taki zosta� osi�gni�ty dla liczby iteracji r�wnej 75 i wsp�czynnika parowania r�wnego 0.4. Nieco gorzej poradzi� sobie algorytm A* z wynikiem 29965 metr�w w czasie 7.91 sekund. Trasa zosta�a znaleziona na etapie pierwszym ze wsp�czynnikiem b��du r�wnym 10 metr�w. Najlepszy osi�gni�ty wynik przez algorytm genetyczny to 30939 metr�w w czasie 8.33 sekund dla liczby populacji r�wnej 2000, liczby iteracji 1000, krzy�owania $PMX$ oraz mutacji $Insert$.

Druga trasa o pocz�tkowej d�ugo�ci 40402 metr�w zosta�a najlepiej skr�cona przez algorytm mr�wkowy do d�ugo�ci 24347 metr�w w czasie 28.32 sekund. Liczba iteracji by�a r�wna 75, a wsp�czynnik parowania by� r�wny 0.4. Algorytm genetyczny znalaz� najlepsze rozwi�zanie o d�ugo�ci 27381 metr�w w czasie 9.03 sekund. Wynik ten zosta� osi�gni�ty dla parametr�w wej�ciowych: liczba populacji r�wna 2000, liczba iteracji r�wna 2000, krzy�owanie $CX$ oraz mutacja$Swap$. Natomiast algorytm A* z rodziny algorytm�w zach�annych wyznaczy� tym razem tras� o d�ugo�ci 28757 metr�w w czasie 8.05 sekund. Trasa zosta�a zbudowana na trzecim etapie ze wsp�czynnikiem b��du 10 metr�w.

Ostatnia trasa o oryginalnej d�ugo�ci wynosz�cej 53478 metr�w zosta�a najlepiej poprawiona przez algorytm mr�wkowy. D�ugo�� poprawionej trasy wynios�a 28172 metr�w, a czas potrzebny na wygenerowanie jest r�wny 131.14 sekund. Taki wynik zosta� osi�gni�ty przy 75 iteracjach oraz wsp�czynniku parowania r�wnego 0.1. Dla algorytm�w zach�annych najlepiej poradzi� sobie algorytm najbli�szego s�siada, wyznaczaj�c tras� o d�ugo�ci 31781 metr�w w czasie 65.1 sekund na etapie pierwszym ze wsp�czynnikiem b��du r�wnym 20 metr�w. Najlepszy wynik dla algorytmu genetycznego wyni�s� 32374 metr�w. Zosta� on osi�gni�ty w czasie 8.25 sekund dla parametr�w wej�ciowych: liczba populacji r�wna 2000, liczba iteracji r�wna 1000, krzy�owanie $CX$ oraz mutacja $Insert$.


Przeprowadzone badania pokazuj� jak mo�na rozwi�za� rzeczywisty problem, oraz polepszy� rozwi�zania w spos�b, jaki nie by� nawet przewidziany przez zleceniodawc�. W dalszej pracy rozwi�zania mog� by� rozwijane w wielu kierunkach, na przyk�ad mo�na zastosowa� podej�cie hybrydowe. Pocz�tkowe trasy do algorytmu genetycznego i mr�wkowego, mog�yby by� generowane za pomoc� algorytm�w zach�annych. Na wszystkich znalezionych trasach mo�na zastosowa� operatory optymalizuj�ce na przyk�ad 2-opt lub inne.



\nocite{*}
\bibliographystyle{plain}
\bibliography{bibliografia}
\addcontentsline{toc}{chapter}{Bibliografia}

% \listoffigures
{%
    \let\oldnumberline\numberline%
    \renewcommand{\numberline}{\tablename~\oldnumberline}%
    \listoftables
}
\addcontentsline{toc}{chapter}{Spis tabel}
% \listoffigures
{
    \let\oldnumberline\numberline%
    \renewcommand{\numberline}{\figurename~\oldnumberline}%
    \listoffigures
}
\addcontentsline{toc}{chapter}{Spis rysunk�w}
\lstlistoflistings
\addcontentsline{toc}{chapter}{Spis listing�w}
\raggedbottom
\listofalgorithms
\addcontentsline{toc}{chapter}{Spis algorytm�w}

\end{document}
