\chapter{Algorytm genetyczny}
	

\section{Chromosom}
	Chromosom z definicji jest to ci�g gen�w reprezentuj�cy dane rozwi�zanie. Z kolei gen przenosi informacj� o cechach. Mo�liwo�� osi�gni�cia sukcesu jest tylko wtedy, gdy odpowiednio zakoduje si� informacj� i ustali funkcj� przystosowania. Do zakodowania badanego problemu u�yj� metody permutacyjnej, gdzie ka�dy punkt musi zosta� odwiedzony tylko raz. Ka�demu punktowi przed wylosowanie tras zostanie przypisany unikalny indeks, b�dzie on odpowiada� genowi. Nast�pnie dla ka�dego z $K$ osobnik�w zostanie zapisany chromosom w postaci ci�gu permutacyjnego. Mo�e to by� $[ 1 2 3 4 5 6 7]$ jak r�wnie� $[6 1 2 4 5 7 3]$. Te dwa chromosomy odwiedzaj� wszystkie punkty w r�nej kolejno�ci oraz tylko raz. 
	
\section{Selekcja}
	W algorytmie genetycznym przed wykonaniem operator�w nale�y dokona� selekcji rodzic�w. Istnieje wiele r�nych sposob�w, ja przebadam dwie najpopularniejsze. Jest to metoda ruletki oraz turniejowa
	
\subsection{Metoda ruletki}
	